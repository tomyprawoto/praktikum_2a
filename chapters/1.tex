
\section{Dwi Yulianingsih}
\subsection{Sejarah Phyton}
Phyton adalah sebuah bahasa pemrograman dengan perancangan yang berfokus pada tingkat keterbacaan kode, menggabungkan kapabilitas, kemampuan dan sintaks kode yang sangat jelas. Phyton juga dilengkapi dengan fungsi pustaka atau library standar yang besar dan didukung oleh komunitas yang besar. Phyton dibuat oleh seseorang keturunan belanda yaitu Guido Van Rossum, awalnya pembuatan phyton ini digunakan untuk pembuatan bahasa tingkat tinggi pada sebuah sistem operasi. Phyton telah digunakan oleh perusahaan-perusahaan untuk membuat perangkat lunak komersil. Pemrograman bahasa python merupakan pemrogram gratis atau freeware, sehingga bisa dikembangkan, dan tidak memiliki batasan dalam peng-copy-an dan didistribusikan. Terdapat beberapa layanan yang diberikan dalam phyton lengkap dengan source kodenya, debugger dan profiler, antarmuka, fungsi sistem, GUI, dan database-nya. Python dapat digunakan untuk berbagai Sistem Operasi, yang diantaranya Unix (linux), PCs (DOS, Windows, OS/2), Machintosh dan sebagainya.

\subsection{Instalasi Anaconda}
\begin{enumerate}
    \item Kita harus menyiapkan instalasi anaconda, kita dapat mendownload nya melalui internet.
    \item Kemudian kita bisa mengklik installer yang telah kita miliki dan tunggu.
    \begin{figure}[!htbp]
        \centering
        \includegraphics[width=3cm,height=3cm]{figures/1.png}
        \caption{gambar1}
        \label{awal}
        \end{figure}

    \item Lalu pada tampilan seperti gambar di bawah klik next.
    \begin{figure}[!htbp]
        \centering
        \includegraphics[width=3cm,height=3cm]{figures/2.png}
        \caption{gambar2}
        \label{next}
        \end{figure}

    \item Setelah itu setujui lisensi yang ada dengan mengklik I Agree
    \begin{figure}[!htbp]
        \centering
        \includegraphics[width=3cm,height=3cm]{figures/3.png}
        \caption{gambar3}
        \label{lisensi}
        \end{figure}

    \item Tunggu instalasi selesai, lalu klik skip
    \begin{figure}[!htbp]
        \centering
        \includegraphics[width=3cm,height=3cm]{figures/4.png}
        \caption{gambar4}
        \label{skip}
        \end{figure}

    \item setelah itu klik finish, dan selesai yeay
    \begin{figure}[!htbp]
        \centering
        \includegraphics[width=3cm,height=3cm]{figures/5.png}
        \caption{gambar5}
        \label{selesai}
        \end{figure}
\end{enumerate}

\subsection{Menggunakan Spyder}
berikut adalah contoh dalam menggunakan spyder
\begin{figure}[!htbp]
    \centering
    \includegraphics[width=3cm,height=3cm]{figures/6.png}
    \caption{gambar6}
    \label{spyder}
    \end{figure}
%%%%%%%%%%%%%%%%%%%%%%%%%%%%%%%%%%%%%%%%%%%%%%%%%%%%%%%	
\section{Dwi Septiani Tsaniyah}
\subsection{Sejarah Python}

Python dikembangkan oleh Guido van Rossum sebagai bahasa pemrograman ABC pada tahun 1990 di Stichting Mathematisch Centrum (CWI) di Amsterdam. Versi terbaru yang dirilis oleh CWI adalah 1.2.

Pada 1995, Guido pindah ke CNRI di Virginia, AS, dan terus mengembangkan Python. Versi terakhir yang dirilis 1.6. Pada tahun 2000, insinyur Guido dan Python menjadi perusahaan komersial untuk BeOpen.com dan menciptakan BeOpen PythonLabs. Python 2.0 dirilis oleh BeOpen. Setelah menghapus Python 2.0, beberapa anggota Guido dan PythonLabs pindah ke DigitalCreations.

Saat ini, pengembangan Python sedang dilanjutkan oleh sekelompok programmer yang dikoordinir oleh Guido dan Yayasan Perangkat Lunak Python. Python Software Foundation, versi 2.1, memiliki hak cipta Python, dan Python adalah organisasi nirlaba yang memblokir kepemilikan perusahaan komersial. Saat ini, distribusi Python telah mencapai versi 2.7.14 dan versi 3.6.3

Program televisi Guido Monty Python Flying Circus telah dinamai nama Python oleh Guido sebagai bahasa ciptaannya. Oleh karena itu, sering kali ungkapan khas suatu acara sering muncul dalam korespondensi antara pengguna Python.

\subsection{Instalasi Anaconda}
\begin{enumerate}
    \item Pastikan Bahwa Python telah terinstal dilaptop anda.
    \item Jika anda belum punya anaconda
    \item Kemudian buka installer yang telah di download barusan
    \item Klik next


    \item Kemudian Klik I Agree


    \item Kemudian pilih akan di instal untuk siapa, kemudian pilih next


    \item Kemudian tentukan dicretory nya


    \item Kemudian Centang yang register Anaconda as default Python, Kemudian Pilih Next


    \item Tunggu Proses Instalasi hingga selesai


    \item Instalasi telah selesai
	\end{enumerate}

\subsection{Penggunaan Spyder}
kodingan sederhana Hello Word

%%%%%%%%%%%%%%%%%%%%%%%%%%%%%%%%%%%%%%%%%%%%%%%%%%%%%%%%%%%%%%%%%%

\section{Kadek Diva Krishna Murti}

\subsection{Sejarah Python}
Python merupakan salah satu bahasa pemrograman tingkat tinggi yang menggunakan metode pemrosesan \textit{interpreted}, dimana kode program akan diproses baris per baris secara langsung dari kode program.

Bahasa pemrograman Python dirilis pertama kali oleh Guido van Rossum di Scitchting Mathematisch Centrum (CWI) Belanda pada tahun 1991. Bahasa python terinspirasi dari bahasa pemrograman ABC. Nama python tidak berasal dari nama ular yang kita kenal. Guido merupakan penggemar grup komedi Inggris bernama Monty Python. Kemudian, ia menamakan Bahasa pemrograman ciptaannya dengan nama Python.

Pada tahun 1994, Python 1.0 dirilis, yang diikuti dengan Python 2.0 pada tahun 2000. Python 3.0 keluar pada tahun 2008. Sampai saat ini Python masih dikembangkan oleh \textit{Python Software Foundation}. Bahasa Python mendukung hampir semua sistem operasi, bahkan untuk sistem operasi Linux, hampir semua distronya sudah menyertakan Python di dalamnya \cite{roihan2017monitoring}.


\subsection{Instalasi Anaconda}
Berikut ini merupakan langkah-langkah cara instalasi Anaconda di windows:
\begin{enumerate}
	\item Pastikan kalian telah menginstall Python sebelumnya.
	\item Klik dua kali pada installer Anaconda. Installer anaconda bisa anda dapatkan di https://www.anaconda.com/distribution/
	\item Setelah itu akan muncul window installernya. Kemudian klik ''Next'' untuk memulai instalasi.
	\begin{figure}[H]
		\includegraphics[width=10cm]{figures/diva/1chp1diva.png}
		\centering
	\end{figure}

	\item Baca Lisensi Agreement Anacondanya. Lalu klik ''I Agree'' jika kalian menerimanya dan untuk melajutkannya instalasinya.
	\begin{figure}[H]
		\includegraphics[width=10cm]{figures/diva/2chp1diva.png}
		\centering
	\end{figure}

	\item Selanjutnya diberi pilihan untuk menginstallnya, apakah hanya untuk kalian atau untuk semua pengguna. Disini saya memilih ''All Users'', lalu klik ''Next''.
	\begin{figure}[H]
		\includegraphics[width=10cm]{figures/diva/3chp1diva.png}
		\centering
	\end{figure}

	\item Kemudian pilih tujuan instalasinya. Disini saya biarkan default folder instalasinya. Setelah itu, klik ''Next''.
	\begin{figure}[H]
		\includegraphics[width=10cm]{figures/diva/4chp1diva.png}
		\centering
	\end{figure}

	\item Setelah itu, kalian diberi beberapa opsi tambahan. Opsi pertama yaitu, ''Add Anaconda to my PATH environment variable''. Opsi ini akan menambahkan Anaconda ke PATH sistem environment variable. Opsi kedua yaitu, ''Register Anaconda as my default Python 3.7''. Opsi ini akan mendaftarkan Anaconda sebagai system Python 3.7. Saya centang kedua opsi tersebut, lalu klik ''Install''.
	\begin{figure}[H]
		\includegraphics[width=10cm]{figures/diva/5chp1diva.png}
		\centering
	\end{figure}

	\item Tunggu hingga proses instalasi selesai.
	\begin{figure}[H]
		\includegraphics[width=10cm]{figures/diva/6chp1diva.png}
		\centering
	\end{figure}

	\item Kemudian klik ''Next'' untuk melanjutkan.
	\begin{figure}[H]
		\includegraphics[width=10cm]{figures/diva/7chp1diva.png}
		\centering
	\end{figure}

	\item Selanjutnya kalian diberi pilihan untuk menginstall Microsoft VSCode. Saya klik ''Skip'' untuk melanjutkan.
	\begin{figure}[H]
		\includegraphics[width=10cm]{figures/diva/8chp1diva.png}
		\centering
	\end{figure}

	\item Kemudian klik ''Finish'' untuk selesai.
	\begin{figure}[H]
		\includegraphics[width=10cm]{figures/diva/9chp1diva.png}
		\centering
	\end{figure}

	\item Untuk mengecek apakah Anaconda telah terinstall yaitu dengan cara membuka Command Prompt. Lalu ketikan ''conda -V'' dan tekan enter, kode itu akan mengecek versi Anaconda yang terinstall.
	\begin{figure}[H]
		\includegraphics[width=10cm]{figures/diva/10chp1diva.png}
		\centering
	\end{figure}

\end{enumerate}

\subsection{Penggunaan Spyder}

Terdapat 2 cara menjalankan Spyder. Yang pertama dengan Anaconda Prompt dan yang kedua dengan Anaconda Navigation. Berikut ini merupakan langkah-langkah cara menjalankan Spyder di windows:

\begin{itemize}
	\item Anaconda Prompt

	\begin{enumerate}
		
		\item Pertama klik start, lalu cari ''Anaconda Prompt''.
		\begin{figure}[H]
			\includegraphics[width=10cm]{figures/diva/11chp1diva.png}
			\centering
		\end{figure}		
		\item Selanjutnya akan muncul sebuah prompt. Kemudian ketikan ''start spyder'' dan tekan enter.
		\begin{figure}[H]
			\includegraphics[width=10cm]{figures/diva/12chp1diva.png}
			\centering
		\end{figure}
		\item Lalu tunggu sampai selesai.
		\begin{figure}[H]
			\includegraphics[width=10cm]{figures/diva/13chp1diva.png}
			\centering
		\end{figure}

	\end{enumerate}
	\item Anaconda Navigation

	\begin{enumerate}
		\item Pertama klik start, lalu cari ''Anaconda Navigation''.
		\begin{figure}[H]
			\includegraphics[width=10cm]{figures/diva/14chp1diva.png}
			\centering
		\end{figure}
		\item Selanjutnya akan muncul sebuah window. Kemudian klik ''Launch'' pada menu Spyder.
		\begin{figure}[H]
			\includegraphics[width=10cm]{figures/diva/15chp1diva.png}
			\centering
		\end{figure}
		\item Lalu tunggu sampai selesai.
		\begin{figure}[H]
			\includegraphics[width=10cm]{figures/diva/16chp1diva.png}
			\centering
		\end{figure}

	\end{enumerate}
\end{itemize}

Apabila muncul window in ketika pertama kali menjalankan Spyder, pilih “Allow Access”.
\begin{figure}[H]
	\includegraphics[width=10cm]{figures/diva/17chp1diva.png}
	\centering
\end{figure}
		
Berikut ini merupakan gambar dari Spyder
\begin{figure}[H]
	\includegraphics[width=10cm]{figures/diva/18chp1diva.png}
	\centering
\end{figure}

Berikut cara menggunakan Spyder:
\begin{enumerate}
	\item Silahkan ketikan script Python anda di sini.
	\begin{figure}[H]
		\includegraphics[width=10cm]{figures/diva/20chp1diva.png}
		\centering
	\end{figure}
	\item Setelah mengetik script Python, kemudian klik tombol play atau tekan tombol F5 untuk mengeksekusi script Python yang telah diketik tadi.
	\begin{figure}[H]
		\includegraphics[width=2cm]{figures/diva/22chp1diva.png}
		\centering
	\end{figure}
	\item Hasil dari eksekusi akan muncul disini.
	\begin{figure}[H]
		\includegraphics[width=10cm]{figures/diva/21chp1diva.png}
		\centering
	\end{figure}
		\item Berikut tampilan penuhnya.
	\begin{figure}[H]
		\includegraphics[width=10cm]{figures/diva/19chp1diva.png}
		\centering
	\end{figure}
\end{enumerate}

%%%%%%%%%%%%%%%%%%%%%%%%%%%%%%%%%%%%%%%%

\section{Harun Ar-Rasyid}
\subsection{Sejarah}
Python diciptakan oleh Guido van Rossum pertama kali di Scitchting Mathematisch Centrum (CWI) di Belanda pada awal tahun 1990-an. Bahasa python terinspirasi dari bahasa pemrograman ABC. Sampai sekarang, Guido masih menjadi penulis utama untuk python, meskipun bersifat open source sehingga ribuan orang juga berkontribusi dalam mengembangkannya.

Pada 1995, Guido pindah ke CNRI di Virginia Amerika sambil terus mengembangkan Python. Versi terakhir yang dirilis adalah 1.6. Pada tahun 2000, pengembang inti Guido dan Python pindah ke BeOpen.com yang merupakan perusahaan komersial dan membentuk BeOpen PythonLabs. Python 2.0 dirilis oleh BeOpen. Setelah menghapus Python 2.0, Guido dan beberapa anggota tim PythonLabs pindah ke DigitalCreations.

Saat ini pengembangan Python terus dilakukan oleh sekelompok programmer yang dikoordinir oleh Guido dan Python Software Foundation. Python Software Foundation adalah organisasi nirlaba yang dibentuk sebagai pemegang hak cipta intelektual Python sejak versi 2.1 dan dengan demikian mencegah Python dimiliki oleh perusahaan komersial. Saat ini distribusi Python telah mencapai versi 2.7.14 dan versi 3.6.3

Nama Python dipilih oleh Guido sebagai nama bahasa ciptaannya karena kecintaan Guido pada acara televisi Flying Circus Monty Python. Oleh karena itu sering ekspresi khas acara sering muncul dalam korespondensi antara pengguna Python.
\subsection{Instalasi Anaconda}
\begin{enumerate}
    \item Pastikan Bahwa Python telah terinstal dilaptop anda.
    \item Kemudian Download Anaconda pada websitenya langsung.
    \item Kemudian buka installer yang telah di download barusan
    \item Klik next
    \begin{figure}[!Htbp]
        \centering
        \includegraphics[width=3cm,height=3cm]{figures/Screenshot(80).png}
        \caption{Tampilan Awal}
        \label{awal}
        \end{figure}

    \item Kemudian Klik I Agree
    \begin{figure}[!Htbp]
        \centering
        \includegraphics[width=3cm,height=3cm]{figures/Screenshot(81).png}
        \caption{License Agreement}
        \label{License}
        \end{figure}

    \item Kemudian pilih akan di instal untuk siapa, kemudian pilih next
    \begin{figure}[!Htbp]
        \centering
        \includegraphics[width=3cm,height=3cm]{figures/Screenshot(82).png}
        \caption{Pemilihan User}
        \label{User}
        \end{figure}

    \item Kemudian tentukan dicretory nya
    \begin{figure}[!Htbp]
        \centering
        \includegraphics[width=3cm,height=3cm]{figures/Screenshot(83).png}
        \caption{Pemilihan Direktori Penyimpanan}
        \label{Directory}
        \end{figure}

    \item Kemudian Centang yang register Anaconda as default Python, Kemudian Pilih Next
    \begin{figure}[!Htbp]
        \centering
        \includegraphics[width=3cm,height=3cm]{figures/Screenshot(84).jpeg}
        \caption{Pemilihan Opsi}
        \label{opsi}
        \end{figure}

    \item Tunggu Proses Instalasi hingga selesai
    \begin{figure}[!Htbp]
        \centering
        \includegraphics[width=3cm,height=3cm]{figures/Screenshot(85).png}
        \caption{Proses Instal}
        \label{Proses}
        \end{figure}

    \item Klik next
    \begin{figure}[!Htbp]
        \centering
        \includegraphics[width=3cm,height=3cm]{figures/Screenshot(86).png}
        \caption{Proses Instal Selesai}
        \label{Proses}
        \end{figure}

    \item kemudian jika kalian belum instal MS VSC di sarankan menginstalnya dlu, jika sudah klik skip
    \begin{figure}[!Htbp]
        \centering
        \includegraphics[width=3cm,height=3cm]{figures/Screenshot(87).png}
        \caption{Penawaran Instal MS VSC}
        \label{offering}
        \end{figure}

    \item Instalasi anaconda telah selesai
    \begin{figure}[!Htbp]
        \centering
        \includegraphics[width=3cm,height=3cm]{figures/Screenshot(88).png}
        \caption{Instalasi Selesai}
        \label{akhir}
        \end{figure}
\end{enumerate}
\subsection{Menggunakan Spyder}
Setelah selesai melakukan instalasi anaconda, maka ada beberapa tool yang digunakan seperti spyder

\begin{figure}[!Htbp]
    \centering
    \includegraphics[width=3cm,height=3cm]{figures/Spyder.png}
    \caption{Ini adalah tampilan spyder}
    \label{spyder}
    \end{figure}

Gambar diatas menjelaskan tentang tampilan spider dan mengexsekusi program halo world.

%%%%%%%%%%%%%%%%%%%%%%%%%%%%%%%%%%%%%%%%%%%%

\section{Rahmatul Ridha / 1144124}
\subsection{Sejarah Python}
Bahasa pemrograman Python adalah bahasa yang dibuat oleh seorang keturunan Belanda yaitu Guido van Rossum. Sampai saat ini Python masih dikembangkan oleh \textit{Python Software Foundation}. Awalnya, pembuatan bahasa pemrograman ini adalah untuk membuat skrip bahasa tingkat tinggi pada sebuah sistem operasi yang terdistribusi Amoeba. Python telah digunakan oleh beberapa pengembang dan bahkan digunakan oleh beberapa perusahaan untuk pembuatan perangkat lunak komersial. Pemrograman bahasa python ini adalah pemrogram gratis atau \textit{freeware}, sehingga dapat dikembangkan, dan tidak ada batasan dalam penyalinannya dan mendistribusikan.

Python dikembangkan oleh Guido van Rossum pada akhir tahun delapan puluhan dan awal tahun sembilan puluhan di National Research Institute for Mathematics and Computer Science di Belanda. Python berasal dari banyak bahasa lain, termasuk ABC, Modula-3, C, C ++, Algol-68, SmallTalk, dan shell Unix dan bahasa script lainnya.
Fitur overview terbaik adalah IT mendukung metode pemrograman fungsional dan terstruktur serta OOP. Hal ini dapat digunakan sebagai bahasa scripting atau dapat dikompilasi untuk byte-kode untuk membangun aplikasi besar. Ini memberikan tingkat yang sangat tinggi pada tipe data dinamis dan mendukung memeriksa jenis dinamis. IT mendukung pengumpulan sampah otomatis. Hal ini dapat dengan mudah diintegrasikan dengan C, C ++, COM, ActiveX, CORBA, dan Java. Hal tersebut menjadi terpopuler karena kemudahan bagi programmer yang menjadikan python pemograman terbaik pada tahun 2016.

Saat ini pengembangan Python terus dilakukan oleh sekumpulan pemrogram yang dikoordinir Guido dan Python Software Foundation. Python Software Foundation adalah sebuah organisasi non-profit yang dibentuk sebagai pemegang hak cipta intelektual Python sejak versi 2.1 dan dengan demikian mencegah Python dimiliki oleh perusahaan komersial. Saat ini distribusi Python sudah mencapai versi 2.7.14 dan versi 3.6.3.

\subsection{Instalasi Anaconda}
\begin{itemize}
\item Instalasi Anaconda
Berikut adalah langkah-langkah cara menginstal Anaconda di Windows:
\begin{enumerate}
\item Download installer anaconda terbaru, seperti pada gambar \ref{download_anaconda}. Kalian dapat memilih versi 2 atau 3, dengan versi Anaconda berapa.
\begin{figure}[ht]
	\centerline{\includegraphics[width=0.70\textwidth]{figures/Rahma/DownloadAnaconda.JPG}}
	\caption{Download Anaconda}
	\label{download_anaconda}
\end{figure}	
\item Setelah selesai mendownload, klik 2 kali pada installer Anaconda.
\item Kemudian akan tampil seperti gambar \ref{gambar1}, lalu klik next.
\begin{figure}[ht]
	\centerline{\includegraphics[width=0.70\textwidth]{figures/Rahma/a.JPG}}
	\caption{Proses 1 }
	\label{gambar1 }
\end{figure}
\item Setelah itu read lisensi dan klik “I Agree”.
\begin{figure}[ht]
	\centerline{\includegraphics[width=0.70\textwidth]{figures/Rahma/b.JPG}}
	\caption{proses 2 }
	\label{gambar2 }
\end{figure}

\item Selanjutnya ada pilihan untuk menginstallnya, yaitu “just me” atau “all users”. Lalu klik next.
\begin{figure}[ht]
	\centerline{\includegraphics[width=0.70\textwidth]{figures/Rahma/c.JPG}}
	\caption{Proses 3 }
	\label{gambar3}
\end{figure}

\item Kemudian pilih okasi yang diinginkan, lalu klik next.
\begin{figure}[ht]
	\centerline{\includegraphics[width=0.70\textwidth]{figures/Rahma/d.JPG}}
	\caption{Proses 4 }
	\label{gambar4 }
\end{figure}

\item Pilih ‘add anaconda to PATH’ atau tidak. Disini kalian memilih apakah akan mendaftarkan Anaconda sebagai default Python 3.7. kacuali kalian berencana menginstal dan menjalankan beberapa versi Anaconda, atau beberapa versi Python, biarkan default dan biarkan kotaknya dicentang. Kemudian klik tombol Install. Jika kalian ingin melihat packages Anaconda yang sedang dipasang, klik Show Details.
\begin{figure}[ht]
	\centerline{\includegraphics[width=0.70\textwidth]{figures/Rahma/e.JPG}}
	\caption{Proses 5 }
	\label{gambar5 }
\end{figure}

\item Jika instalasi selesai, kemudian klik next.
\begin{figure}[ht]
	\centerline{\includegraphics[width=0.70\textwidth]{figures/Rahma/f.JPG}}
	\caption{Proses 6 }
	\label{gambar6 }
\end{figure}

\item Jika packages telah selesai diinstall, maka aka nada perintah untuk menginstall VS Code, lalu klik tombol Install Microsoft VS Code.
\begin{figure}[ht]
	\centerline{\includegraphics[width=0.70\textwidth]{figures/Rahma/g.JPG}}
	\caption{Proses 7}
	\label{gambar7 }
\end{figure}

\item Setelah instalasi selesai, maka akan terlihat kotal dialog “Thanks for Installing Anaconda3”. Lalu klik Finish.
\begin{figure}[ht]
	\centerline{\includegraphics[width=0.70\textwidth]{figures/Rahma/h.JPG}}
	\caption{Proses 8}
	\label{gambar8 }
\end{figure}
	    \end{enumerate}
\end{itemize}

%%%%%%%%%%%%%%%%%%%%%%%%%%%%%%%%%%%%%%%%%%%%
