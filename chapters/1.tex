\section{Dwi Yulianingsih}
\subsection{Sejarah Phyton}
Phyton adalah sebuah bahasa pemrograman dengan perancangan yang berfokus pada tingkat keterbacaan kode, menggabungkan kapabilitas, kemampuan dan sintaks kode yang sangat jelas. Phyton juga dilengkapi dengan fungsi pustaka atau library standar yang besar dan didukung oleh komunitas yang besar. Phyton dibuat oleh seseorang keturunan belanda yaitu Guido Van Rossum, awalnya pembuatan phyton ini digunakan untuk pembuatan bahasa tingkat tinggi pada sebuah sistem operasi. Phyton telah digunakan oleh perusahaan-perusahaan untuk membuat perangkat lunak komersil. Pemrograman bahasa python merupakan pemrogram gratis atau freeware, sehingga bisa dikembangkan, dan tidak memiliki batasan dalam peng-copy-an dan didistribusikan. Terdapat beberapa layanan yang diberikan dalam phyton lengkap dengan source kodenya, debugger dan profiler, antarmuka, fungsi sistem, GUI, dan database-nya. Python dapat digunakan untuk berbagai Sistem Operasi, yang diantaranya Unix (linux), PCs (DOS, Windows, OS/2), Machintosh dan sebagainya.

\subsection{Instalasi Anaconda}
\begin{enumerate}
    \item Kita harus menyiapkan instalasi anaconda, kita dapat mendownload nya melalui internet.
    \item Kemudian kita bisa mengklik installer yang telah kita miliki dan tunggu.
    \begin{figure}[!htbp]
        \centering
        \includegraphics[width=3cm,height=3cm]{figures/1.png}
        \caption{gambar1}
        \label{awal}
        \end{figure}

    \item Lalu pada tampilan seperti gambar di bawah klik next.
    \begin{figure}[!htbp]
        \centering
        \includegraphics[width=3cm,height=3cm]{figures/2.png}
        \caption{gambar2}
        \label{next}
        \end{figure}

    \item Setelah itu setujui lisensi yang ada dengan mengklik I Agree
    \begin{figure}[!htbp]
        \centering
        \includegraphics[width=3cm,height=3cm]{figures/3.png}
        \caption{gambar3}
        \label{lisensi}
        \end{figure}

    \item Tunggu instalasi selesai, lalu klik skip
    \begin{figure}[!htbp]
        \centering
        \includegraphics[width=3cm,height=3cm]{figures/4.png}
        \caption{gambar4}
        \label{skip}
        \end{figure}

    \item setelah itu klik finish, dan selesai yeay
    \begin{figure}[!htbp]
        \centering
        \includegraphics[width=3cm,height=3cm]{figures/5.png}
        \caption{gambar5}
        \label{selesai}
        \end{figure}
\end{enumerate}

\subsection{Menggunakan Spyder}
berikut adalah contoh dalam menggunakan spyder
\begin{figure}[!htbp]
    \centering
    \includegraphics[width=3cm,height=3cm]{figures/6.png}
    \caption{gambar6}
    \label{spyder}
    \end{figure}
%%%%%%%%%%%%%%%%%%%%%%%%%%%%%%%%%%%%%%%%%%%%%%%%%%%%%%%	
\section{Dwi Septiani Tsaniyah}
\subsection{Sejarah Python}

Python dikembangkan oleh Guido van Rossum sebagai bahasa pemrograman ABC pada tahun 1990 di Stichting Mathematisch Centrum (CWI) di Amsterdam. Versi terbaru yang dirilis oleh CWI adalah 1.2.

Pada 1995, Guido pindah ke CNRI di Virginia, AS, dan terus mengembangkan Python. Versi terakhir yang dirilis 1.6. Pada tahun 2000, insinyur Guido dan Python menjadi perusahaan komersial untuk BeOpen.com dan menciptakan BeOpen PythonLabs. Python 2.0 dirilis oleh BeOpen. Setelah menghapus Python 2.0, beberapa anggota Guido dan PythonLabs pindah ke DigitalCreations.

Saat ini, pengembangan Python sedang dilanjutkan oleh sekelompok programmer yang dikoordinir oleh Guido dan Yayasan Perangkat Lunak Python. Python Software Foundation, versi 2.1, memiliki hak cipta Python, dan Python adalah organisasi nirlaba yang memblokir kepemilikan perusahaan komersial. Saat ini, distribusi Python telah mencapai versi 2.7.14 dan versi 3.6.3

Program televisi Guido Monty Python Flying Circus telah dinamai nama Python oleh Guido sebagai bahasa ciptaannya. Oleh karena itu, sering kali ungkapan khas suatu acara sering muncul dalam korespondensi antara pengguna Python.

\subsection{Instalasi Anaconda}
\begin{enumerate}
    \item Pastikan Bahwa Python telah terinstal dilaptop anda.
    \item Jika anda belum punya anaconda
    \item Kemudian buka installer yang telah di download barusan
    \item Klik next


    \item Kemudian Klik I Agree


    \item Kemudian pilih akan di instal untuk siapa, kemudian pilih next


    \item Kemudian tentukan dicretory nya


    \item Kemudian Centang yang register Anaconda as default Python, Kemudian Pilih Next


    \item Tunggu Proses Instalasi hingga selesai


    \item Instalasi telah selesai
	\end{enumerate}

\subsection{Penggunaan Spyder}
kodingan sederhana Hello Word

%%%%%%%%%%%%%%%%%%%%%%%%%%%%%%%%%%%%%%%%%%%%%%%%%%%%%%%%%%%%%%%%%%

\section{Kadek Diva Krishna Murti}

\subsection{Sejarah Python}
Python merupakan salah satu bahasa pemrograman tingkat tinggi yang menggunakan metode pemrosesan \textit{interpreted}, dimana kode program akan diproses baris per baris secara langsung dari kode program.

Bahasa pemrograman Python dirilis pertama kali oleh Guido van Rossum di Scitchting Mathematisch Centrum (CWI) Belanda pada tahun 1991. Bahasa python terinspirasi dari bahasa pemrograman ABC. Nama python tidak berasal dari nama ular yang kita kenal. Guido merupakan penggemar grup komedi Inggris bernama Monty Python. Kemudian, ia menamakan Bahasa pemrograman ciptaannya dengan nama Python.

Pada tahun 1994, Python 1.0 dirilis, yang diikuti dengan Python 2.0 pada tahun 2000. Python 3.0 keluar pada tahun 2008. Sampai saat ini Python masih dikembangkan oleh \textit{Python Software Foundation}. Bahasa Python mendukung hampir semua sistem operasi, bahkan untuk sistem operasi Linux, hampir semua distronya sudah menyertakan Python di dalamnya \cite{roihan2017monitoring}.


\subsection{Instalasi Anaconda}
Berikut ini merupakan langkah-langkah cara instalasi Anaconda di windows:
\begin{enumerate}
	\item Pastikan kalian telah menginstall Python sebelumnya.
	\item Klik dua kali pada installer Anaconda. Installer anaconda bisa anda dapatkan di https://www.anaconda.com/distribution/
	\item Setelah itu akan muncul window installernya. Kemudian klik ''Next'' untuk memulai instalasi.
	\begin{figure}[H]
		\includegraphics[width=10cm]{figures/diva/1chp1diva.png}
		\centering
	\end{figure}

	\item Baca Lisensi Agreement Anacondanya. Lalu klik ''I Agree'' jika kalian menerimanya dan untuk melajutkannya instalasinya.
	\begin{figure}[H]
		\includegraphics[width=10cm]{figures/diva/2chp1diva.png}
		\centering
	\end{figure}

	\item Selanjutnya diberi pilihan untuk menginstallnya, apakah hanya untuk kalian atau untuk semua pengguna. Disini saya memilih ''All Users'', lalu klik ''Next''.
	\begin{figure}[H]
		\includegraphics[width=10cm]{figures/diva/3chp1diva.png}
		\centering
	\end{figure}

	\item Kemudian pilih tujuan instalasinya. Disini saya biarkan default folder instalasinya. Setelah itu, klik ''Next''.
	\begin{figure}[H]
		\includegraphics[width=10cm]{figures/diva/4chp1diva.png}
		\centering
	\end{figure}

	\item Setelah itu, kalian diberi beberapa opsi tambahan. Opsi pertama yaitu, ''Add Anaconda to my PATH environment variable''. Opsi ini akan menambahkan Anaconda ke PATH sistem environment variable. Opsi kedua yaitu, ''Register Anaconda as my default Python 3.7''. Opsi ini akan mendaftarkan Anaconda sebagai system Python 3.7. Saya centang kedua opsi tersebut, lalu klik ''Install''.
	\begin{figure}[H]
		\includegraphics[width=10cm]{figures/diva/5chp1diva.png}
		\centering
	\end{figure}

	\item Tunggu hingga proses instalasi selesai.
	\begin{figure}[H]
		\includegraphics[width=10cm]{figures/diva/6chp1diva.png}
		\centering
	\end{figure}

	\item Kemudian klik ''Next'' untuk melanjutkan.
	\begin{figure}[H]
		\includegraphics[width=10cm]{figures/diva/7chp1diva.png}
		\centering
	\end{figure}

	\item Selanjutnya kalian diberi pilihan untuk menginstall Microsoft VSCode. Saya klik ''Skip'' untuk melanjutkan.
	\begin{figure}[H]
		\includegraphics[width=10cm]{figures/diva/8chp1diva.png}
		\centering
	\end{figure}

	\item Kemudian klik ''Finish'' untuk selesai.
	\begin{figure}[H]
		\includegraphics[width=10cm]{figures/diva/9chp1diva.png}
		\centering
	\end{figure}

	\item Untuk mengecek apakah Anaconda telah terinstall yaitu dengan cara membuka Command Prompt. Lalu ketikan ''conda -V'' dan tekan enter, kode itu akan mengecek versi Anaconda yang terinstall.
	\begin{figure}[H]
		\includegraphics[width=10cm]{figures/diva/10chp1diva.png}
		\centering
	\end{figure}

\end{enumerate}

\subsection{Penggunaan Spyder}

Terdapat 2 cara menjalankan Spyder. Yang pertama dengan Anaconda Prompt dan yang kedua dengan Anaconda Navigation. Berikut ini merupakan langkah-langkah cara menjalankan Spyder di windows:

\begin{itemize}
	\item Anaconda Prompt

	\begin{enumerate}
		
		\item Pertama klik start, lalu cari ''Anaconda Prompt''.
		\begin{figure}[H]
			\includegraphics[width=10cm]{figures/diva/11chp1diva.png}
			\centering
		\end{figure}		
		\item Selanjutnya akan muncul sebuah prompt. Kemudian ketikan ''start spyder'' dan tekan enter.
		\begin{figure}[H]
			\includegraphics[width=10cm]{figures/diva/12chp1diva.png}
			\centering
		\end{figure}
		\item Lalu tunggu sampai selesai.
		\begin{figure}[H]
			\includegraphics[width=10cm]{figures/diva/13chp1diva.png}
			\centering
		\end{figure}

	\end{enumerate}
	\item Anaconda Navigation

	\begin{enumerate}
		\item Pertama klik start, lalu cari ''Anaconda Navigation''.
		\begin{figure}[H]
			\includegraphics[width=10cm]{figures/diva/14chp1diva.png}
			\centering
		\end{figure}
		\item Selanjutnya akan muncul sebuah window. Kemudian klik ''Launch'' pada menu Spyder.
		\begin{figure}[H]
			\includegraphics[width=10cm]{figures/diva/15chp1diva.png}
			\centering
		\end{figure}
		\item Lalu tunggu sampai selesai.
		\begin{figure}[H]
			\includegraphics[width=10cm]{figures/diva/16chp1diva.png}
			\centering
		\end{figure}

	\end{enumerate}
\end{itemize}

Apabila muncul window in ketika pertama kali menjalankan Spyder, pilih “Allow Access”.
\begin{figure}[H]
	\includegraphics[width=10cm]{figures/diva/17chp1diva.png}
	\centering
\end{figure}
		
Berikut ini merupakan gambar dari Spyder
\begin{figure}[H]
	\includegraphics[width=10cm]{figures/diva/18chp1diva.png}
	\centering
\end{figure}

Berikut cara menggunakan Spyder:
\begin{enumerate}
	\item Silahkan ketikan script Python anda di sini.
	\begin{figure}[H]
		\includegraphics[width=10cm]{figures/diva/20chp1diva.png}
		\centering
	\end{figure}
	\item Setelah mengetik script Python, kemudian klik tombol play atau tekan tombol F5 untuk mengeksekusi script Python yang telah diketik tadi.
	\begin{figure}[H]
		\includegraphics[width=2cm]{figures/diva/22chp1diva.png}
		\centering
	\end{figure}
	\item Hasil dari eksekusi akan muncul disini.
	\begin{figure}[H]
		\includegraphics[width=10cm]{figures/diva/21chp1diva.png}
		\centering
	\end{figure}
		\item Berikut tampilan penuhnya.
	\begin{figure}[H]
		\includegraphics[width=10cm]{figures/diva/19chp1diva.png}
		\centering
	\end{figure}
\end{enumerate}

%%%%%%%%%%%%%%%%%%%%%%%%%%%%%%%%%%%%%%%%

\section{Harun Ar-Rasyid}
\subsection{Sejarah}
Python diciptakan oleh Guido van Rossum pertama kali di Scitchting Mathematisch Centrum (CWI) di Belanda pada awal tahun 1990-an. Bahasa python terinspirasi dari bahasa pemrograman ABC. Sampai sekarang, Guido masih menjadi penulis utama untuk python, meskipun bersifat open source sehingga ribuan orang juga berkontribusi dalam mengembangkannya.

Pada 1995, Guido pindah ke CNRI di Virginia Amerika sambil terus mengembangkan Python. Versi terakhir yang dirilis adalah 1.6. Pada tahun 2000, pengembang inti Guido dan Python pindah ke BeOpen.com yang merupakan perusahaan komersial dan membentuk BeOpen PythonLabs. Python 2.0 dirilis oleh BeOpen. Setelah menghapus Python 2.0, Guido dan beberapa anggota tim PythonLabs pindah ke DigitalCreations.

Saat ini pengembangan Python terus dilakukan oleh sekelompok programmer yang dikoordinir oleh Guido dan Python Software Foundation. Python Software Foundation adalah organisasi nirlaba yang dibentuk sebagai pemegang hak cipta intelektual Python sejak versi 2.1 dan dengan demikian mencegah Python dimiliki oleh perusahaan komersial. Saat ini distribusi Python telah mencapai versi 2.7.14 dan versi 3.6.3

Nama Python dipilih oleh Guido sebagai nama bahasa ciptaannya karena kecintaan Guido pada acara televisi Flying Circus Monty Python. Oleh karena itu sering ekspresi khas acara sering muncul dalam korespondensi antara pengguna Python.
\subsection{Instalasi Anaconda}
\begin{enumerate}
    \item Pastikan Bahwa Python telah terinstal dilaptop anda.
    \item Kemudian Download Anaconda pada websitenya langsung.
    \item Kemudian buka installer yang telah di download barusan
    \item Klik next
    \begin{figure}[!Htbp]
        \centering
        \includegraphics[width=3cm,height=3cm]{figures/Screenshot(80).png}
        \caption{Tampilan Awal}
        \label{awal}
        \end{figure}

    \item Kemudian Klik I Agree
    \begin{figure}[!Htbp]
        \centering
        \includegraphics[width=3cm,height=3cm]{figures/Screenshot(81).png}
        \caption{License Agreement}
        \label{License}
        \end{figure}

    \item Kemudian pilih akan di instal untuk siapa, kemudian pilih next
    \begin{figure}[!Htbp]
        \centering
        \includegraphics[width=3cm,height=3cm]{figures/Screenshot(82).png}
        \caption{Pemilihan User}
        \label{User}
        \end{figure}

    \item Kemudian tentukan dicretory nya
    \begin{figure}[!Htbp]
        \centering
        \includegraphics[width=3cm,height=3cm]{figures/Screenshot(83).png}
        \caption{Pemilihan Direktori Penyimpanan}
        \label{Directory}
        \end{figure}

    \item Kemudian Centang yang register Anaconda as default Python, Kemudian Pilih Next
    \begin{figure}[!Htbp]
        \centering
        \includegraphics[width=3cm,height=3cm]{figures/Screenshot(84).jpeg}
        \caption{Pemilihan Opsi}
        \label{opsi}
        \end{figure}

    \item Tunggu Proses Instalasi hingga selesai
    \begin{figure}[!Htbp]
        \centering
        \includegraphics[width=3cm,height=3cm]{figures/Screenshot(85).png}
        \caption{Proses Instal}
        \label{Proses}
        \end{figure}

    \item Klik next
    \begin{figure}[!Htbp]
        \centering
        \includegraphics[width=3cm,height=3cm]{figures/Screenshot(86).png}
        \caption{Proses Instal Selesai}
        \label{Proses}
        \end{figure}

    \item kemudian jika kalian belum instal MS VSC di sarankan menginstalnya dlu, jika sudah klik skip
    \begin{figure}[!Htbp]
        \centering
        \includegraphics[width=3cm,height=3cm]{figures/Screenshot(87).png}
        \caption{Penawaran Instal MS VSC}
        \label{offering}
        \end{figure}

    \item Instalasi anaconda telah selesai
    \begin{figure}[!Htbp]
        \centering
        \includegraphics[width=3cm,height=3cm]{figures/Screenshot(88).png}
        \caption{Instalasi Selesai}
        \label{akhir}
        \end{figure}
\end{enumerate}
\subsection{Menggunakan Spyder}
Setelah selesai melakukan instalasi anaconda, maka ada beberapa tool yang digunakan seperti spyder

\begin{figure}[!Htbp]
    \centering
    \includegraphics[width=3cm,height=3cm]{figures/Spyder.png}
    \caption{Ini adalah tampilan spyder}
    \label{spyder}
    \end{figure}

Gambar diatas menjelaskan tentang tampilan spider dan mengexsekusi program halo world.

%%%%%%%%%%%%%%%%%%%%%%%%%%%%%%%%%%%%%%%%%%%%

\section{Rahmatul Ridha / 1144124}
\subsection{Sejarah Python}
Bahasa pemrograman Python adalah bahasa yang dibuat oleh seorang keturunan Belanda yaitu Guido van Rossum. Sampai saat ini Python masih dikembangkan oleh \textit{Python Software Foundation}. Awalnya, pembuatan bahasa pemrograman ini adalah untuk membuat skrip bahasa tingkat tinggi pada sebuah sistem operasi yang terdistribusi Amoeba. Python telah digunakan oleh beberapa pengembang dan bahkan digunakan oleh beberapa perusahaan untuk pembuatan perangkat lunak komersial. Pemrograman bahasa python ini adalah pemrogram gratis atau \textit{freeware}, sehingga dapat dikembangkan, dan tidak ada batasan dalam penyalinannya dan mendistribusikan.

Saat ini pengembangan Python terus dilakukan oleh sekumpulan pemrogram yang dikoordinir Guido dan Python Software Foundation. Python Software Foundation adalah sebuah organisasi non-profit yang dibentuk sebagai pemegang hak cipta intelektual Python sejak versi 2.1 dan dengan demikian mencegah Python dimiliki oleh perusahaan komersial. Saat ini distribusi Python sudah mencapai versi 2.7.14 dan versi 3.6.3.

\subsection{Instalasi Anaconda}
\begin{itemize}
\item Instalasi Anaconda
Berikut adalah langkah-langkah cara menginstal Anaconda di Windows:
\begin{enumerate}
\item Download installer anaconda terbaru, seperti pada gambar \ref{downloadanaconda}. Kalian dapat memilih versi 2 atau 3, dengan versi Anaconda berapa.
\begin{figure}[ht]
	\centerline{\includegraphics[width=0.70\textwidth]{figures/Rahma/DownloadAnaconda.JPG}}
	\caption{Download Anaconda}
	\label{downloadanaconda}
\end{figure}	
\item Setelah selesai mendownload, klik 2 kali pada installer Anaconda.
\item Kemudian akan tampil seperti gambar \ref{gambar1}, lalu klik next.
\begin{figure}[ht]
	\centerline{\includegraphics[width=0.70\textwidth]{figures/Rahma/a.JPG}}
	\caption{Proses 1 }
	\label{gambar1}
\end{figure}
\item Setelah itu read lisensi dan klik “I Agree”.
\begin{figure}[ht]
	\centerline{\includegraphics[width=0.70\textwidth]{figures/Rahma/b.JPG}}
	\caption{proses 2 }
	\label{gambar2 }
\end{figure}

\item Selanjutnya ada pilihan untuk menginstallnya, yaitu “just me” atau “all users”. Lalu klik next.
\begin{figure}[ht]
	\centerline{\includegraphics[width=0.70\textwidth]{figures/Rahma/c.JPG}}
	\caption{Proses 3 }
	\label{gambar3}
\end{figure}

\item Kemudian pilih okasi yang diinginkan, lalu klik next.
\begin{figure}[ht]
	\centerline{\includegraphics[width=0.70\textwidth]{figures/Rahma/d.JPG}}
	\caption{Proses 4 }
	\label{gambar4 }
\end{figure}

\item Pilih ‘add anaconda to PATH’ atau tidak. Disini kalian memilih apakah akan mendaftarkan Anaconda sebagai default Python 3.7. kacuali kalian berencana menginstal dan menjalankan beberapa versi Anaconda, atau beberapa versi Python, biarkan default dan biarkan kotaknya dicentang. Kemudian klik tombol Install. Jika kalian ingin melihat packages Anaconda yang sedang dipasang, klik Show Details.
\begin{figure}[ht]
	\centerline{\includegraphics[width=0.70\textwidth]{figures/Rahma/e.JPG}}
	\caption{Proses 5 }
	\label{gambar5 }
\end{figure}

\item Jika instalasi selesai, kemudian klik next.
\begin{figure}[ht]
	\centerline{\includegraphics[width=0.70\textwidth]{figures/Rahma/f.JPG}}
	\caption{Proses 6 }
	\label{gambar6 }
\end{figure}

\item Jika packages telah selesai diinstall, maka aka nada perintah untuk menginstall VS Code, lalu klik tombol Install Microsoft VS Code.
\begin{figure}[ht]
	\centerline{\includegraphics[width=0.70\textwidth]{figures/Rahma/g.JPG}}
	\caption{Proses 7}
	\label{gambar7 }
\end{figure}

\item Setelah instalasi selesai, maka akan terlihat kotal dialog “Thanks for Installing Anaconda3”. Lalu klik Finish.
\begin{figure}[ht]
	\centerline{\includegraphics[width=0.70\textwidth]{figures/Rahma/h.JPG}}
	\caption{Proses 8}
	\label{gambar8 }
\end{figure}
	    \end{enumerate}
\end{itemize}

%%%%%%%%%%%%%%%%%%%%%%%%%%%%%%%%%%%%%%%%%%%%
\section{Dwi Yulianingsih}
\subsection{Sejarah Phyton}
Phyton adalah sebuah bahasa pemrograman dengan perancangan yang berfokus pada tingkat keterbacaan kode, menggabungkan kapabilitas, kemampuan dan sintaks kode yang sangat jelas. Phyton juga dilengkapi dengan fungsi pustaka atau library standar yang besar dan didukung oleh komunitas yang besar. Phyton dibuat oleh seseorang keturunan belanda yaitu Guido Van Rossum, awalnya pembuatan phyton ini digunakan untuk pembuatan bahasa tingkat tinggi pada sebuah sistem operasi. Phyton telah digunakan oleh perusahaan-perusahaan untuk membuat perangkat lunak komersil. Pemrograman bahasa python merupakan pemrogram gratis atau freeware, sehingga bisa dikembangkan, dan tidak memiliki batasan dalam peng-copy-an dan didistribusikan. Terdapat beberapa layanan yang diberikan dalam phyton lengkap dengan source kodenya, debugger dan profiler, antarmuka, fungsi sistem, GUI, dan database-nya. Python dapat digunakan untuk berbagai Sistem Operasi, yang diantaranya Unix (linux), PCs (DOS, Windows, OS/2), Machintosh dan sebagainya.

\subsection{Instalasi Anaconda}
\begin{enumerate}
    \item Kita harus menyiapkan instalasi anaconda, kita dapat mendownload nya melalui internet.
    \item Kemudian kita bisa mengklik installer yang telah kita miliki dan tunggu.
    \begin{figure}[!htbp]
        \centering
        \includegraphics[width=3cm,height=3cm]{figures/1.png}
        \caption{gambar1}
        \label{awal}
        \end{figure}

    \item Lalu pada tampilan seperti gambar di bawah klik next.
    \begin{figure}[!htbp]
        \centering
        \includegraphics[width=3cm,height=3cm]{figures/2.png}
        \caption{gambar2}
        \label{next}
        \end{figure}

    \item Setelah itu setujui lisensi yang ada dengan mengklik I Agree
    \begin{figure}[!htbp]
        \centering
        \includegraphics[width=3cm,height=3cm]{figures/3.png}
        \caption{gambar3}
        \label{lisensi}
        \end{figure}

    \item Tunggu instalasi selesai, lalu klik skip
    \begin{figure}[!htbp]
        \centering
        \includegraphics[width=3cm,height=3cm]{figures/4.png}
        \caption{gambar4}
        \label{skip}
        \end{figure}

    \item setelah itu klik finish, dan selesai yeay
    \begin{figure}[!htbp]
        \centering
        \includegraphics[width=3cm,height=3cm]{figures/5.png}
        \caption{gambar5}
        \label{selesai}
        \end{figure}
\end{enumerate}

\subsection{Menggunakan Spyder}
berikut adalah contoh dalam menggunakan spyder
\begin{figure}[!htbp]
    \centering
    \includegraphics[width=3cm,height=3cm]{figures/6.png}
    \caption{gambar6}
    \label{spyder}
    \end{figure}
%%%%%%%%%%%%%%%%%%%%%%%%%%%%%%%%%%%%%%%%%%%%%%%%%%%%%%%	
\section{Dwi Septiani Tsaniyah}
\subsection{Sejarah Python}

Python dikembangkan oleh Guido van Rossum sebagai bahasa pemrograman ABC pada tahun 1990 di Stichting Mathematisch Centrum (CWI) di Amsterdam. Versi terbaru yang dirilis oleh CWI adalah 1.2.

Pada 1995, Guido pindah ke CNRI di Virginia, AS, dan terus mengembangkan Python. Versi terakhir yang dirilis 1.6. Pada tahun 2000, insinyur Guido dan Python menjadi perusahaan komersial untuk BeOpen.com dan menciptakan BeOpen PythonLabs. Python 2.0 dirilis oleh BeOpen. Setelah menghapus Python 2.0, beberapa anggota Guido dan PythonLabs pindah ke DigitalCreations.

Saat ini, pengembangan Python sedang dilanjutkan oleh sekelompok programmer yang dikoordinir oleh Guido dan Yayasan Perangkat Lunak Python. Python Software Foundation, versi 2.1, memiliki hak cipta Python, dan Python adalah organisasi nirlaba yang memblokir kepemilikan perusahaan komersial. Saat ini, distribusi Python telah mencapai versi 2.7.14 dan versi 3.6.3

Program televisi Guido Monty Python Flying Circus telah dinamai nama Python oleh Guido sebagai bahasa ciptaannya. Oleh karena itu, sering kali ungkapan khas suatu acara sering muncul dalam korespondensi antara pengguna Python.

\subsection{Instalasi Anaconda}
\begin{enumerate}
    \item Pastikan Bahwa Python telah terinstal dilaptop anda.
    \item Jika anda belum punya anaconda
    \item Kemudian buka installer yang telah di download barusan
    \item Klik next


    \item Kemudian Klik I Agree


    \item Kemudian pilih akan di instal untuk siapa, kemudian pilih next


    \item Kemudian tentukan dicretory nya


    \item Kemudian Centang yang register Anaconda as default Python, Kemudian Pilih Next


    \item Tunggu Proses Instalasi hingga selesai


    \item Instalasi telah selesai
	\end{enumerate}

\subsection{Penggunaan Spyder}
kodingan sederhana Hello Word

%%%%%%%%%%%%%%%%%%%%%%%%%%%%%%%%%%%%%%%%%%%%%%%%%%%%%%%%%%%%%%%%%%

\section{Kadek Diva Krishna Murti}

\subsection{Sejarah Python}
Python merupakan salah satu bahasa pemrograman tingkat tinggi yang menggunakan metode pemrosesan \textit{interpreted}, dimana kode program akan diproses baris per baris secara langsung dari kode program.

Bahasa pemrograman Python dirilis pertama kali oleh Guido van Rossum di Scitchting Mathematisch Centrum (CWI) Belanda pada tahun 1991. Bahasa python terinspirasi dari bahasa pemrograman ABC. Nama python tidak berasal dari nama ular yang kita kenal. Guido merupakan penggemar grup komedi Inggris bernama Monty Python. Kemudian, ia menamakan Bahasa pemrograman ciptaannya dengan nama Python.

Pada tahun 1994, Python 1.0 dirilis, yang diikuti dengan Python 2.0 pada tahun 2000. Python 3.0 keluar pada tahun 2008. Sampai saat ini Python masih dikembangkan oleh \textit{Python Software Foundation}. Bahasa Python mendukung hampir semua sistem operasi, bahkan untuk sistem operasi Linux, hampir semua distronya sudah menyertakan Python di dalamnya \cite{roihan2017monitoring}.


\subsection{Instalasi Anaconda}
Berikut ini merupakan langkah-langkah cara instalasi Anaconda di windows:
\begin{enumerate}
	\item Pastikan kalian telah menginstall Python sebelumnya.
	\item Klik dua kali pada installer Anaconda. Installer anaconda bisa anda dapatkan di https://www.anaconda.com/distribution/
	\item Setelah itu akan muncul window installernya. Kemudian klik ''Next'' untuk memulai instalasi.
	\begin{figure}[H]
		\includegraphics[width=10cm]{figures/diva/1chp1diva.png}
		\centering
	\end{figure}

	\item Baca Lisensi Agreement Anacondanya. Lalu klik ''I Agree'' jika kalian menerimanya dan untuk melajutkannya instalasinya.
	\begin{figure}[H]
		\includegraphics[width=10cm]{figures/diva/2chp1diva.png}
		\centering
	\end{figure}

	\item Selanjutnya diberi pilihan untuk menginstallnya, apakah hanya untuk kalian atau untuk semua pengguna. Disini saya memilih ''All Users'', lalu klik ''Next''.
	\begin{figure}[H]
		\includegraphics[width=10cm]{figures/diva/3chp1diva.png}
		\centering
	\end{figure}

	\item Kemudian pilih tujuan instalasinya. Disini saya biarkan default folder instalasinya. Setelah itu, klik ''Next''.
	\begin{figure}[H]
		\includegraphics[width=10cm]{figures/diva/4chp1diva.png}
		\centering
	\end{figure}

	\item Setelah itu, kalian diberi beberapa opsi tambahan. Opsi pertama yaitu, ''Add Anaconda to my PATH environment variable''. Opsi ini akan menambahkan Anaconda ke PATH sistem environment variable. Opsi kedua yaitu, ''Register Anaconda as my default Python 3.7''. Opsi ini akan mendaftarkan Anaconda sebagai system Python 3.7. Saya centang kedua opsi tersebut, lalu klik ''Install''.
	\begin{figure}[H]
		\includegraphics[width=10cm]{figures/diva/5chp1diva.png}
		\centering
	\end{figure}

	\item Tunggu hingga proses instalasi selesai.
	\begin{figure}[H]
		\includegraphics[width=10cm]{figures/diva/6chp1diva.png}
		\centering
	\end{figure}

	\item Kemudian klik ''Next'' untuk melanjutkan.
	\begin{figure}[H]
		\includegraphics[width=10cm]{figures/diva/7chp1diva.png}
		\centering
	\end{figure}

	\item Selanjutnya kalian diberi pilihan untuk menginstall Microsoft VSCode. Saya klik ''Skip'' untuk melanjutkan.
	\begin{figure}[H]
		\includegraphics[width=10cm]{figures/diva/8chp1diva.png}
		\centering
	\end{figure}

	\item Kemudian klik ''Finish'' untuk selesai.
	\begin{figure}[H]
		\includegraphics[width=10cm]{figures/diva/9chp1diva.png}
		\centering
	\end{figure}

	\item Untuk mengecek apakah Anaconda telah terinstall yaitu dengan cara membuka Command Prompt. Lalu ketikan ''conda -V'' dan tekan enter, kode itu akan mengecek versi Anaconda yang terinstall.
	\begin{figure}[H]
		\includegraphics[width=10cm]{figures/diva/10chp1diva.png}
		\centering
	\end{figure}

\end{enumerate}

\subsection{Penggunaan Spyder}

Terdapat 2 cara menjalankan Spyder. Yang pertama dengan Anaconda Prompt dan yang kedua dengan Anaconda Navigation. Berikut ini merupakan langkah-langkah cara menjalankan Spyder di windows:

\begin{itemize}
	\item Anaconda Prompt

	\begin{enumerate}
		
		\item Pertama klik start, lalu cari ''Anaconda Prompt''.
		\begin{figure}[H]
			\includegraphics[width=10cm]{figures/diva/11chp1diva.png}
			\centering
		\end{figure}		
		\item Selanjutnya akan muncul sebuah prompt. Kemudian ketikan ''start spyder'' dan tekan enter.
		\begin{figure}[H]
			\includegraphics[width=10cm]{figures/diva/12chp1diva.png}
			\centering
		\end{figure}
		\item Lalu tunggu sampai selesai.
		\begin{figure}[H]
			\includegraphics[width=10cm]{figures/diva/13chp1diva.png}
			\centering
		\end{figure}

	\end{enumerate}
	\item Anaconda Navigation

	\begin{enumerate}
		\item Pertama klik start, lalu cari ''Anaconda Navigation''.
		\begin{figure}[H]
			\includegraphics[width=10cm]{figures/diva/14chp1diva.png}
			\centering
		\end{figure}
		\item Selanjutnya akan muncul sebuah window. Kemudian klik ''Launch'' pada menu Spyder.
		\begin{figure}[H]
			\includegraphics[width=10cm]{figures/diva/15chp1diva.png}
			\centering
		\end{figure}
		\item Lalu tunggu sampai selesai.
		\begin{figure}[H]
			\includegraphics[width=10cm]{figures/diva/16chp1diva.png}
			\centering
		\end{figure}

	\end{enumerate}
\end{itemize}

Apabila muncul window in ketika pertama kali menjalankan Spyder, pilih “Allow Access”.
\begin{figure}[H]
	\includegraphics[width=10cm]{figures/diva/17chp1diva.png}
	\centering
\end{figure}
		
Berikut ini merupakan gambar dari Spyder
\begin{figure}[H]
	\includegraphics[width=10cm]{figures/diva/18chp1diva.png}
	\centering
\end{figure}

Berikut cara menggunakan Spyder:
\begin{enumerate}
	\item Silahkan ketikan script Python anda di sini.
	\begin{figure}[H]
		\includegraphics[width=10cm]{figures/diva/20chp1diva.png}
		\centering
	\end{figure}
	\item Setelah mengetik script Python, kemudian klik tombol play atau tekan tombol F5 untuk mengeksekusi script Python yang telah diketik tadi.
	\begin{figure}[H]
		\includegraphics[width=2cm]{figures/diva/22chp1diva.png}
		\centering
	\end{figure}
	\item Hasil dari eksekusi akan muncul disini.
	\begin{figure}[H]
		\includegraphics[width=10cm]{figures/diva/21chp1diva.png}
		\centering
	\end{figure}
		\item Berikut tampilan penuhnya.
	\begin{figure}[H]
		\includegraphics[width=10cm]{figures/diva/19chp1diva.png}
		\centering
	\end{figure}
\end{enumerate}

%%%%%%%%%%%%%%%%%%%%%%%%%%%%%%%%%%%%%%%%

\section{Harun Ar-Rasyid}
\subsection{Sejarah}
Python diciptakan oleh Guido van Rossum pertama kali di Scitchting Mathematisch Centrum (CWI) di Belanda pada awal tahun 1990-an. Bahasa python terinspirasi dari bahasa pemrograman ABC. Sampai sekarang, Guido masih menjadi penulis utama untuk python, meskipun bersifat open source sehingga ribuan orang juga berkontribusi dalam mengembangkannya.

Pada 1995, Guido pindah ke CNRI di Virginia Amerika sambil terus mengembangkan Python. Versi terakhir yang dirilis adalah 1.6. Pada tahun 2000, pengembang inti Guido dan Python pindah ke BeOpen.com yang merupakan perusahaan komersial dan membentuk BeOpen PythonLabs. Python 2.0 dirilis oleh BeOpen. Setelah menghapus Python 2.0, Guido dan beberapa anggota tim PythonLabs pindah ke DigitalCreations.

Saat ini pengembangan Python terus dilakukan oleh sekelompok programmer yang dikoordinir oleh Guido dan Python Software Foundation. Python Software Foundation adalah organisasi nirlaba yang dibentuk sebagai pemegang hak cipta intelektual Python sejak versi 2.1 dan dengan demikian mencegah Python dimiliki oleh perusahaan komersial. Saat ini distribusi Python telah mencapai versi 2.7.14 dan versi 3.6.3

Nama Python dipilih oleh Guido sebagai nama bahasa ciptaannya karena kecintaan Guido pada acara televisi Flying Circus Monty Python. Oleh karena itu sering ekspresi khas acara sering muncul dalam korespondensi antara pengguna Python.
\subsection{Instalasi Anaconda}
\begin{enumerate}
    \item Pastikan Bahwa Python telah terinstal dilaptop anda.
    \item Kemudian Download Anaconda pada websitenya langsung.
    \item Kemudian buka installer yang telah di download barusan
    \item Klik next
    \begin{figure}[!Htbp]
        \centering
        \includegraphics[width=3cm,height=3cm]{figures/Screenshot(80).png}
        \caption{Tampilan Awal}
        \label{awal}
        \end{figure}

    \item Kemudian Klik I Agree
    \begin{figure}[!Htbp]
        \centering
        \includegraphics[width=3cm,height=3cm]{figures/Screenshot(81).png}
        \caption{License Agreement}
        \label{License}
        \end{figure}

    \item Kemudian pilih akan di instal untuk siapa, kemudian pilih next
    \begin{figure}[!Htbp]
        \centering
        \includegraphics[width=3cm,height=3cm]{figures/Screenshot(82).png}
        \caption{Pemilihan User}
        \label{User}
        \end{figure}

    \item Kemudian tentukan dicretory nya
    \begin{figure}[!Htbp]
        \centering
        \includegraphics[width=3cm,height=3cm]{figures/Screenshot(83).png}
        \caption{Pemilihan Direktori Penyimpanan}
        \label{Directory}
        \end{figure}

    \item Kemudian Centang yang register Anaconda as default Python, Kemudian Pilih Next
    \begin{figure}[!Htbp]
        \centering
        \includegraphics[width=3cm,height=3cm]{figures/Screenshot(84).jpeg}
        \caption{Pemilihan Opsi}
        \label{opsi}
        \end{figure}

    \item Tunggu Proses Instalasi hingga selesai
    \begin{figure}[!Htbp]
        \centering
        \includegraphics[width=3cm,height=3cm]{figures/Screenshot(85).png}
        \caption{Proses Instal}
        \label{Proses}
        \end{figure}

    \item Klik next
    \begin{figure}[!Htbp]
        \centering
        \includegraphics[width=3cm,height=3cm]{figures/Screenshot(86).png}
        \caption{Proses Instal Selesai}
        \label{Proses}
        \end{figure}

    \item kemudian jika kalian belum instal MS VSC di sarankan menginstalnya dlu, jika sudah klik skip
    \begin{figure}[!Htbp]
        \centering
        \includegraphics[width=3cm,height=3cm]{figures/Screenshot(87).png}
        \caption{Penawaran Instal MS VSC}
        \label{offering}
        \end{figure}

    \item Instalasi anaconda telah selesai
    \begin{figure}[!Htbp]
        \centering
        \includegraphics[width=3cm,height=3cm]{figures/Screenshot(88).png}
        \caption{Instalasi Selesai}
        \label{akhir}
        \end{figure}
\end{enumerate}
\subsection{Menggunakan Spyder}
Setelah selesai melakukan instalasi anaconda, maka ada beberapa tool yang digunakan seperti spyder

\begin{figure}[!Htbp]
    \centering
    \includegraphics[width=3cm,height=3cm]{figures/Spyder.png}
    \caption{Ini adalah tampilan spyder}
    \label{spyder}
    \end{figure}

Gambar diatas menjelaskan tentang tampilan spider dan mengexsekusi program halo world.

%%%%%%%%%%%%%%%%%%%%%%%%%%%%%%%%%%%%%%%%%%%%

\section{Felix Lase}
\subsection{sejarah}
Python diciptakan oleh Guido van Rossum untuk pertama kalinya di Scitchting Mathematisch Centrum (CWI) di Belanda pada awal 1990-an. Bahasa Python terinspirasi oleh bahasa pemrograman ABC. Sampai sekarang, Guido masih menjadi penulis utama untuk Python, meskipun open source terbuka untuk ribuan orang yang juga berkontribusi pada pengembangannya.
\par
Pada tahun 1995, Guido terus membuat python di Corporate for National Research Initiative (CNRI) di Virginia America, tempat ia merilis beberapa versi Python.
\par
Pada bulan Mei 2000, Guido dan tim Python pindah ke BeOpen.com dan membentuk tim BeOpen PythonLabs. Pada bulan Oktober tahun yang sama, tim Python pindah ke Digital Creation (sekarang Zope Company). Pada tahun 2001, Organisasi Python dibentuk, Yayasan Perangkat Lunak Python (PSF). PSF adalah organisasi nirlaba yang khusus dibuat untuk semua hal yang berkaitan dengan kekayaan intelektual Python. Perusahaan Zope adalah anggota sponsor PSF.
\par
Semua versi Python yang dirilis adalah open source. Dalam sejarahnya, hampir semua rilis python menggunakan lisensi yang kompatibel dengan GFL. Berikutnya adalah versi minor dari walikota dan python bersama dengan tanggal rilis.Instalasi anaconda
\subsection{Instalasi Anaconda}
\begin{enumerate}
    \item Terlebih dahulu kita harus mendownload python, sebelum anaconda diinstal
    \item Buka installer klik Next
     \begin{figure}[!htbp]
        \centering
        \includegraphics[width=3cm,height=3cm]{figures/felix/1.png}
        \caption{Tampilan Awal}
        \label{awal}
        \end{figure}
    \item Klik I Agree untuk membuka lisensi
     \begin{figure}[!htbp]
        \centering
        \includegraphics[width=3cm,height=3cm]{figures/felix/2.png}
        \caption{License}
        \label{awal}
        \end{figure}
    \item  Pilih untuk siapa aplikasi diinstal bisa just me dan juga bisa all users
     \begin{figure}[!htbp]
        \centering
        \includegraphics[width=3cm,height=3cm]{figures/felix/3.png}
        \caption{Proses}
        \label{awal}
        \end{figure}
    \item  Pilih lokasi instalasi
    \begin{figure}[!htbp]
        \centering
        \includegraphics[width=3cm,height=3cm]{figures/felix/4.png}
        \caption{Proses}
        \label{awal}
        \end{figure}
    \item Pilih register anaconda karna add aconda environment tidak remomended
    \begin{figure}[!htbp]
        \centering
        \includegraphics[width=3cm,height=3cm]{figures/felix/5.png}
        \caption{Proses}
        \label{awal}
        \end{figure}
    \item Tunggu hingga selesai
    \begin{figure}[!htbp]
        \centering
        \includegraphics[width=3cm,height=3cm]{figures/felix/6.png}
        \caption{Proses}
        \label{awal}
        \end{figure}
    \item Klik skip
    \begin{figure}[!htbp]
        \centering
        \includegraphics[width=3cm,height=3cm]{figures/felix/8.png}
        \caption{Proses}
        \label{awal}
        \end{figure}
    \item  Dan anaconda berhasil di install
    \begin{figure}[!htbp]
        \centering
        \includegraphics[width=3cm,height=3cm]{figures/felix/9.png}
        \caption{Proses}
        \label{awal}
        \end{figure}
\end{enumerate}
\subsection{Menggunakan Spyder}
setelah selesai menggunakan instalasi anaconda,  maka ada beberapa tool yang digunakan seperti spyder
\begin{figure}[!htbp]
        \centering
        \includegraphics[width=3cm,height=3cm]{figures/felix/10.png}
        \caption{Proses}
        \label{awal}
        \end{figure}

Gambar tersebut menjelaskan tentang tampilan spyder dan mengeksekusi program aa

%%%%%%%%%%%%%%%%%%%%%%%%%%%%%%%%%%%%%%%%%%%%%%%%%%%%%%%%%%%


\section{Oniwaldus Bere Mali}
\subsection{Sejarah Python}
              Python dikembangkan oleh Guido van Rossum pada tahun 1990 di CWI, Amsterdam sebagai kelanjutan dari bahasa pemrograman ABC. Tahun 1995, Guido pindah ke CNRI sambil terus melanjutkan pengembangan Python. Versi terakhir yang dikeluarkan adalah 1.6. Tahun 2000, Guido dan para pengembang inti Python pindah ke BeOpen.com yang merupakan sebuah perusahaan komersial dan membentuk BeOpen PythonLabs. Python 2.0 dikeluarkan oleh BeOpen. Setelah mengeluarkan Python 2.0, Guido dan beberapa anggota tim PythonLabs pindah ke DigitalCreations. Saat ini pengembangan Python terus dilakukan oleh sekumpulan pemrogram yang dikoordinir Guido dan Python Software Foundation. Python Software Foundation adalah sebuah organisasi non-profit yang dibentuk sebagai pemegang hak cipta intelektual Python sejak versi 2.1 dan dengan demikian mencegah Python dimiliki oleh perusahaan komersial. Saat ini distribusi Python sudah mencapai versi 2.6.1 dan versi 3.0. Nama Python dipilih oleh Guido sebagai nama bahasa ciptaannya karena kecintaan guido pada acara televisi Monty Python's Flying Circus. Oleh karena itu seringkali ungkapan-ungkapan khas dari acara tersebut seringkali muncul dalam korespondensi antar pengguna Python.

\subsection{Instalasi Anaconda}
\begin{enumerate}
    \item Pastikan Bahwa Python telah terinstal dilaptop anda.
    \item Jika anda belum punya anaconda, kalian bisa download
    \item Kemudian buka installer yang telah di download barusan
    \item Klik next
    \begin{figure}[H]
        \centering
        \includegraphics[width=3cm,height=3cm]{figures/oni/1.png}
        \label{awal}
        \end{figure}

    \item klik next
    \begin{figure}[H]
        \centering
        \includegraphics[width=3cm,height=3cm]{figures/oni/2.png}
        \caption{Klik Next}
        \label{License}
        \end{figure}

    \item Klik pada I Agree
     \begin{figure}[H]
        \centering
        \includegraphics[width=3cm,height=3cm]{figures/oni/3.png}
        \caption{Klik pada I Agreer}
        \label{User}
        \end{figure}

    \begin{figure}[!htbp]
        \centering
        \includegraphics[width=3cm,height=3cm]{figures/oni/4.png}
        \label{Directory}
        \end{figure}

     \begin{figure}[H]
        \centering
        \includegraphics[width=3cm,height=3cm]{figures/oni/5.png}
        \label{opsi}
        \end{figure}

    \item Ceklis pada Add Anaconda to my PATH environment varable dan Register Anaconda as my default Python 3.7 .selanjutnya klik Install
    \begin{figure}[H]
        \centering
        \includegraphics[width=3cm,height=3cm]{figures/oni/6.png}
        \caption{Ceklis pada Add Anaconda to my PATH environment varable dan Register Anaconda as my default Python 3.7 .selanjutnya klik Install}
        \label{Proses}
        \end{figure}

    \item Klik next
    \begin{figure}[H]
        \centering
        \includegraphics[width=3cm,height=3cm]{figures/oni/7.png}
        \caption{Proses Instali}
        \label{Proses}
        \end{figure}

    \item klik next
    \begin{figure}[H]
        \centering
        \includegraphics[width=3cm,height=3cm]{figures/oni/8.png}
        \caption{klik next}
        \label{offering}
        \end{figure}

 \item instal MS VSC di sarankan menginstalnya dulu
    \begin{figure}[H]
        \centering
        \includegraphics[width=3cm,height=3cm]{figures/oni/9.png}
        \caption{Klik install microsoft vscode}
        \label{offering}
        \end{figure}

    \item Instalasi anaconda telah selesai
     \begin{figure}[H]
        \centering
        \includegraphics[width=3cm,height=3cm]{figures/oni/10.png}
        \caption{Instalasi Selesai}
        \label{akhir}
        \end{figure}
\end{enumerate}

\subsection{Menggunakan Spyder}

 \begin{figure}[H]
        \centering
        \includegraphics[width=3cm,height=3cm]{figures/oni/oni.png}
        \label{akhir}
        \end{figure}

     \begin{figure}[H]
        \centering
        \includegraphics[width=3cm,height=3cm]{figures/oni/11.png}
        \label{akhir}
        \end{figure}

%%%%%%%%%%%%%%%%%%%%%%%%%%%%%%%%%%%%%%%%%%%%%%


\section{DezhaAidiMartha}

\subsection{Sejarah}
Python merupakan bahasa pemograman yang diciptakan oleh Guido vanRossum. Hingga sekarang, bahasa pemograman Python ini masih dikembangkan oleh \textit{Python Software Foundation}. Pada awalnya bahasa pemograman ini dikembangkan untuk membuat skrip bahasa pemograman tingkat tinggi pada suatu sistem oeperasi yang terdistribusi oleh Amoeba. Python ini telah digunakan oleh beberapa pengembang dan bahkan telah digunakan oleh beberapa perusahaan untuk pembuatan perangkat lunak komersial. Dan lebihnya lagi bahasa pemograman Python ini gratis atau \textit{freewere}, sehingga dapat dikembangkan, dan tidak ada batasan dalam penyalinannya.

Bahasa pemograman python dikembangkan oleh Guido van Rossum pada akhir tahun 80-an dan awal tahun 90-an di National Research Institude for Mathematics and Computer Science di Belanda. Python berasal dari banyak bahasa latin, termasuk ABC, Modula-3, C, C++, Algol-68, SmallTalk, dan shell Unix dan bahasa script lainnya.

Saat ini pengembangan Python terus dilakukan oleh sekompulan programmer yang di koordinir oleh Guido dan Python Software Foundation. Python Software Foundation adalah sebuah organisasi non-profit yang dibentuk sebagai pemegang hak cipta intelektual Python sejak versi 2.1 dan mencegah Python dimiliki oleh perusahaan komersial. Saat ini distribusi Python sudah mencapai versi 3.7.2.

\subsection{InstalasiAnaconda}
\begin{enumerate}
	\item Pastikan bahwa ython telah terinstall.
	\begin{figure}[H]
		\centering
		\includegraphics[width=3cm,height=3cm]{figures/dezha/Screenshot1.png}
		\label{awal}
		\end{figure}
	\item Jika anda belum punya anaconda,kalian bisa download.
	\item Kemudian buka installer yang telah di downlaod.
	\item Klik next
	\begin{figure}[H]
		\centering
		\includegraphics[width=3cm,height=3cm]{figures/dezha/Screenshot2.png}
		\label{Awal}
		\end{figure}

	\item Klik I Agree
	\begin{figure}[H]
		\centering
		\includegraphics[width=3cm,height=3cm]{figures/dezha/Screenshot3.png}
		\label{Klik I Agree}
		\end{figure}

	\item Klik next
	\begin{figure}[H]
		\centering
		\includegraphics[width=3cm,height=3cm]{figures/dezha/Screenshot4.png}
		\label{Pilih Just me, dan next}
		\end{figure}

	\item Klik next
	\begin{figure}[H]
		\centering
		\includegraphics[width=3cm,height=3cm]{figures/dezha/Screenshot5.png}
		\label{Memilih letak dictionary}
		\end{figure}

	\item Klik Install
	\begin{figure}[H]
		\centering
		\includegraphics[width=3cm,height=3cm]{figures/dezha/Screenshot7.png}
		\label{Pilih Register Anaconda as my default Python 3.2}
		\end{figure}

	\item Tunggu hingga selesai
	\begin{figure}[H]
		\centering
		\includegraphics[width=3cm,height=3cm]{figures/dezha/Screenshot9.png}
		\label{Finish}
		\end{figure}

\end{enumerate}
\subsection{Spyder}
\begin{enumerate}
	\item Running Spyder dan lakukan code dibawah ini
	\begin{figure}[H]
		\centering
		\includegraphics[width=3cm,height=3cm]{figures/dezha/Screenshot10.png}
		\label{Finish}
		\end{figure}

\end{enumerate}

%%%%%%%%%%%%%%%%%%%%%%%%%%%%%%%%%%%%%%%%%%%%%%%%%%
\section{Choirul Anam}
\subsection{Sejarah Phyton}
Guido Van Rossum adalah orang yang menciptakan python di Scitching Mathematisch Centrum (CWI) Belanda pada wal tahun 1990-an. Pada bahasa python  sendiri terinspirasi dari salahsatu pemrograman ABC. Sampai sekarangpun Guido Van Rossum masih menjadi orang yang pertama untuk python, walaupun bersifat open source, dengan itu banyak orang bahkan ribuan orang dapat berkontribusi dalam mengembangkannya.
\par
Pada tahun 1995, Di Virginia Amerika Guido Van Rossum melanjutkan mengembangkan python di Corporation for National Reserch Initiative (CNRI), dia merilis beberapa versi pada python.
\par
Pada Mei tahun 2000, Guido Van Rossum dan tim pythonnya pindah ke BeOpen.com dan membentuk tim baru yaitu BeOpen PythonLabs, Pada bulan Oktober di tahun yang sama, timnya pindah ke Digital Creation dan sekarang menjadi perusahaan Zope.
\par
Dibentuklah organisasi python yaitu Python Software Foundation (PSF). Organisasi ini merupakan organisasi nirlaba yang dibentuk khusus untuk semua hal yang bertentangan dengan hak intelektual python. Perusahaan Zope menjadi anggota sponsor dari organisasi PSF (Python Software Foundation).
\par
Pada nama python sendiri berasal dari nama ular yang kita tau. Guido Van Rossum seorang yang menggemarkan grub komedi inggris yang bernama Monty Python, Lalu ia menamakan ciptaannya dengan nama Python.
\par
Untuk versi python yang pernah dirilis semuanya bersifat open source. Python yang dirilis hamper semuanya menggunakan liensi GFL-compatible.
\subsection{Instalasi Anaconda}
\begin{enumerate}
    \item Sebelumnya anda download file installernya di http://repo.continuum.io/archive silahkan anda pilih versi python 2 atau python 3 dan versi anaconda berapa, disini menggunakan anaconda versi Anaconda3 5.1.0 64bit.
    \begin{figure}[!htbp]
        \centering
        \includegraphics[width=3cm,height=3cm]{figures/choi/1.png}
        \caption{gambar1}
        \label{gambar pertama}
        \end{figure}

    \item Jika anda sudah memilih versi yang mana silahkan download dan tunggu proses downloadnya selesai, Setelah di download silahkan ke file download tersebut lalu klik dukali pada file installernya kemudian klik next.
    \item Lalu read lisensi dan klik I Agree, setelah itu akan muncul tampilan seperti berikut dan silahkan klik next.
    \begin{figure}[!htbp]
        \centering
        \includegraphics[width=3cm,height=3cm]{figures/choi/2.png}
        \caption{gambar2}
        \label{gambar kedua}
        \end{figure}

    \item Jika anda sudah klik next lalu pilih Ad anaconda to path lalu aka nada tampilan seperti berikut dan biarkan kotak tersebut tercentang anda klik next saja.
    \begin{figure}[!htbp]
        \centering
        \includegraphics[width=3cm,height=3cm]{figures/choi/3.png}
        \caption{gambar3}
        \label{gambar ketiga}
        \end{figure}
    \item Jika anda sudah klik next lalu klik tombol install, dan jika anda ingin melihat pakages anaconda yang sedang di pasang silahkan anda klik show details.
    \item Selanjutnya anda klik next, lalu untuk mengistal VS codenya silahkan klik Install Microsoft VS code, setelah proses instalasi selesai lalu anda pilih next, jika anda ingin menginstal anaconda tanpa VS codenya pilih skip.
    \begin{figure}[!htbp]
        \centering
        \includegraphics[width=3cm,height=3cm]{figures/choi/4.png}
        \caption{gambar4}
        \label{gambar ke empat}
        \end{figure}

    \item Setelah proses instalasinya selesai maka akan muncul tampilan tersebut dengan penanda bahwa anaconda sudah terinstal di laptop atau computer anda lalu anda pilih Finish.
    \begin{figure}[!htbp]
        \centering
        \includegraphics[width=3cm,height=3cm]{figures/choi/5.png}
        \caption{gambar5}
        \label{gambar kelima}
        \end{figure}
\end{enumerate}

\subsection{Menggunakan Spyder}
Pada penulisan script kode-kode python biasanya di tulis di text editor seperti notepad, sublime dan semacamnya lalu untuk menjalankannya di jalankan melalui cmd. Tapi disini ada yang beda dimana ada text editor untuk kode-kode python dan bisa di jalankan saat klik run saja, text editor tersebut yaitu Spyder.
\par Spyder merupakan text editor dari anaconda dimana di tools ini bisa menjalankan perintah-perintah dengan klik run saja. Di anaconda sendiri mempunyai beberapa text editor untuk python tapi yang akan dijelaskan kali ini menggunakan spyder.
\begin{figure}[!htbp]
    \centering
    \includegraphics[width=3cm,height=3cm]{figures/choi/6.png}
    \caption{gambar6}
    \label{spyder}
    \end{figure}
	

%%%%%%%%%%%%%%%%%%%%%%%%%%%%%%%%%%%%%%%%%%5


\section{Nico Ekklesia Sembiring}
\subsection{sejarah pyton}
Python merupakan bahasa pemrograman yang dikembangkan oleh Guido van Rossum pada tahun akhir tahun 1980 di CWI, Belanda. Kemudian diimplementasikan pada Desember 1989 sebagai kelanjutan dari bahasa pemrograman ABC dan dapat berinteraksi dengan sistem operasi Amuba. Versi terakhir yang dirilis oleh CWI adalah 1.2. Kemudian pada tahun 1995 Guido pindah ke CNRI sambil terus melanjutkan pengembangan Python. Versi terakhir yang dirilis oleh CNRI adalah versi 1.6 yang dirilis pada Tahun 2000.

Selanjutnya Guido dan para pengembang inti Python pindah ke BeOpen.com yang merupakan sebuah perusahaan komersial dan mereka membentuk BeOpen PythonLabs. BeOpen kemudian merilis Python versi 2.0. Setelah merilis Python 2.0, Guido dan beberapa anggota tim PythonLabs pindah ke DigitalCreations.

Nama Python ini dipilih oleh Guido yang merupakan pengembangnya sebagai nama Bahasa pemrograman yang dibuatnya karena kecintaan Guido pada salah satu acara televisi Monty Python’s Flying Circus. Oleh karena itu seringkali ungkapan-ungkapan khas dari acara tersebut muncul dalam korespondensi antar pengguna Python

Hingga saat ini pengembangan Python terus dilakukan oleh sekumpulan pemrogram yang dikoordinir Guido dan Python Software Foundation. Python Software Foundation merupakan sebuah organisasi non-profit yang dibentuk sebagai pemegang hak cipta intelektual Python sejak versi 2.1. Hal ini untuk mencegah Python dimiliki oleh perusahaan komersial. Saat ini Python sudah dirilis hingga versi 2.7.13 dan juga versi 3.6.0.

\subsection{Menginstal Anaconda}
\begin{enumerate}
\item Pastikan terlebih dahulu bahwa Python telah terinstal dilaptop anda.
\item Untuk melakukan instalasi Anaconda, kita harus punya file setup Anaconda nya terlebih dahulu. Kemudian Pilih file nya
\begin{figure}[!htbp]
    \centering
    \includegraphics[width=3cm,height=3cm]{figures/1.png}
    \caption{File Anaconda}
    \label{file}
    \end{figure}

\item Setelah setup terbuka, maka akan mucul informasi seperti berikut. Kemudian klik Next
\begin{figure}[!htbp]
    \centering
    \includegraphics[width=3cm,height=3cm]{figures/2.png}
    \caption{Tampilan Awal}
    \label{awal}
    \end{figure}

\item Setelah itu akan muncul persetujuan lisensi. Setelah membaca lisensi tersebut, kita memilih “I Agree” untuk melanjutkan menginstal Anaconda
\begin{figure}[!htbp]
    \centering
    \includegraphics[width=3cm,height=3cm]{figures/3.png}
    \caption{Persetujuan Lisensi}
    \label{lisensi}
    \end{figure}

\item Setelah menyetui lisensi, kita diharuskan memilih tipe instalasi. Setelah memilih tipe instalasi, maka kita klik “Next”.
\begin{figure}[!htbp]
    \centering
    \includegraphics[width=3cm,height=3cm]{figures/4.png}
    \caption{Tipe Instalasi}
    \label{instalasi}
    \end{figure}

\item Selanjutnya kita diharuskan memilih lokasi untuk diinstal. Setelah memilih lokasi, klik “Next”
\begin{figure}[!htbp]
    \centering
    \includegraphics[width=3cm,height=3cm]{figures/5.png}
    \caption{Pilih Lokasi}
    \label{lokasi}
    \end{figure}

\item Setelah itu centang pada kotak register. Lalu pilih Instal
\begin{figure}[!htbp]
    \centering
    \includegraphics[width=3cm,height=3cm]{figures/6.png}
    \caption{Advanced Options}
    \label{options}
    \end{figure}

\item Setelah install selesai, maka pilih “Next”.
\begin{figure}[!htbp]
    \centering
    \includegraphics[width=3cm,height=3cm]{figures/7.png}
    \caption{Instal Selesai}
    \label{instal}
    \end{figure}

\item Jika tidak akan menginstal VSCode, maka bisa memilih “Skip”
\begin{figure}[!htbp]
    \centering
    \includegraphics[width=3cm,height=3cm]{figures/8.png}
    \caption{Instal VSCode}
    \label{vscode}
    \end{figure}

\item Proses instalasi telah selesai. Kemudian pilih finish.
\begin{figure}[!htbp]
    \centering
    \includegraphics[width=3cm,height=3cm]{figures/9.png}
    \caption{Instalasi Selesai}
    \label{finish}
    \end{figure}
\end{enumerate}

\subsection{Menggunakan Spyder}
Spyder merupakan tool yang tersedia saat anaconda diinstal.
\begin{figure}[!htbp]
    \centering
    \includegraphics[width=3cm,height=3cm]{figures/10.png}
    \caption{Kode}
    \label{code}
    \end{figure}

Gambar merupakan tampilan kode sederhana saat menggunakan spyder. berikut adalah hasilnya
\begin{figure}[!htbp]
    \centering
    \includegraphics[width=3cm,height=3cm]{figures/11.png}
    \caption{Hasil Kode}
    \label{hasil}
    \end{figure}
\section{Damara Benedicta}
\subsection{Sejarah}
Python merupakan suatau bahasa pemrograman skrip yang tidak sulit untuk dibaca dan ditulis, dan tidak memiliki fungsi yang terbatas sehingga dapat digunakan untuk menyelesaikan berbagai macam tugas
Python awal mula dikembangkan tahun 1990 oleh guido van Rossum di Amsterdam. Nama python berasal dari nama yang dipilih oleh guido dari acara televisi sirkus monty pyton.
Pada 1995 versi yang dirilis oleh CWI adalah versi 1.2. dan versi terakhir pada tahun 2000 adalah versi 1.6. dan python 2.0 mulai dirilis oleh BeOpen. Setelah menghapus Python 2.0, Guido dan anggota tim PythonLabs lainnya pindah ke DigitalCreations. Distribusi Python telah mencapai versi 2.6.1 dan versi 3.0, dan dengan demikian mencegah Python dimiliki oleh perusahaan komersial. Saat ini distribusi Python telah mencapai versi 2.7.14 dan versi 3.6.3
\subsection{Instalasi Anaconda}
\begin{enumerate}
    \item Pastikan Bahwa Python telah terinstal dilaptop anda.
    \item Kemudian buka installer yang telah di download barusan
    \item Klik next
    \begin{figure}[!htbp]
        \centering
        \includegraphics[width=3cm,height=3cm]{figures/1.png}
        \caption{Tampilan Awal}
        \label{awal}
        \end{figure}

    \item Kemudian Klik Next
    \begin{figure}[!htbp]
        \centering
        \includegraphics[width=3cm,height=3cm]{figures/2.png}
        \caption{License Agreement}
        \label{License}
        \end{figure}

    \item Kemudian akan muncul persetujuab lisensi, setelah itu pilih " I Agree"
    \begin{figure}[!htbp]
        \centering
        \includegraphics[width=3cm,height=3cm]{figures/3.png}
        \caption{Pemilihan User}
        \label{User}
        \end{figure}

    \item Kemudian memilih tipe instalasi
    \begin{figure}[!htbp]
        \centering
        \includegraphics[width=3cm,height=3cm]{figures/4.png}
        \caption{Pemilihan Direktori Penyimpanan}
        \label{Directory}
        \end{figure}

    \item Kemudian pilih lokasi untuk instal, setelah itu klik Next
    \begin{figure}[!htbp]
        \centering
        \includegraphics[width=3cm,height=3cm]{figures/5.png}
        \caption{Pemilihan Opsi}
        \label{opsi}
        \end{figure}

    \item Kemudian centang pada kotak register dan pilih instal
    \begin{figure}[!htbp]
        \centering
        \includegraphics[width=3cm,height=3cm]{figures/6.png}
        \caption{Proses Instal}
        \label{Proses}
        \end{figure}

    \item Kemudian Klik next
    \begin{figure}[!htbp]
        \centering
        \includegraphics[width=3cm,height=3cm]{figures/7.png}
        \caption{Proses Instal Selesai}
        \label{Proses}
        \end{figure}

    \item kemudian jika tidak akan mengistal VSCode, maka pilih skip
    \begin{figure}[!htbp]
        \centering
        \includegraphics[width=3cm,height=3cm]{figures/8.png}
        \caption{Penawaran Instal MS VSC}
        \label{offering}
        \end{figure}
        \item kemudian Proses intalasi telah selesai pilih finish
    \begin{figure}[!htbp]
        \centering
        \includegraphics[width=3cm,height=3cm]{figures/9.png}
        \caption{Penawaran Instal MS VSC}
        \label{offering}
        \end{figure}
\end{enumerate}
\subsection(Spyder)
Spyder adalah tools yang dipakai untuk python
    \begin{figure}[!htbp]
        \centering
        \includegraphics[width=3cm,height=3cm]{figures/Screenshot(1).png}
        \caption{Instalasi Selesai}
        \label{akhir}
        \end{figure}
%%%%%%%%%%%%%%%%%%%%%%%%%%%%%%%%%%%%%%%%%%%%%%%%%%%%%%%%%%%%%%%%%

\section{Damara Benedicta}
\subsection{Sejarah}
Python merupakan suatau bahasa pemrograman skrip yang tidak sulit untuk dibaca dan ditulis, dan tidak memiliki fungsi yang terbatas sehingga dapat digunakan untuk menyelesaikan berbagai macam tugas
Python awal mula dikembangkan tahun 1990 oleh guido van Rossum di Amsterdam. Nama python berasal dari nama yang dipilih oleh guido dari acara televisi sirkus monty pyton.
Pada 1995 versi yang dirilis oleh CWI adalah versi 1.2. dan versi terakhir pada tahun 2000 adalah versi 1.6. dan python 2.0 mulai dirilis oleh BeOpen. Setelah menghapus Python 2.0, Guido dan anggota tim PythonLabs lainnya pindah ke DigitalCreations. Distribusi Python telah mencapai versi 2.6.1 dan versi 3.0, dan dengan demikian mencegah Python dimiliki oleh perusahaan komersial. Saat ini distribusi Python telah mencapai versi 2.7.14 dan versi 3.6.3
\subsection{Instalasi Anaconda}
\begin{enumerate}
    \item Pastikan Bahwa Python telah terinstal dilaptop anda.
    \item Jika anda belum punya anaconda, kalian bisa download
    \item Kemudian buka installer yang telah di download barusan
    \item Klik next
    \begin{figure}[!htbp]
        \centering
        \includegraphics[width=3cm,height=3cm]{figures/g1.png}
        \caption{Tampilan Awal}
        \label{awal}
        \end{figure}

    \item Kemudian Klik Next
    \begin{figure}[!htbp]
        \centering
        \includegraphics[width=3cm,height=3cm]{figures/g2.png}
        \caption{License Agreement}
        \label{License}
        \end{figure}

    \item Kemudian akan muncul persetujuab lisensi, setelah itu pilih " I Agree"
    \begin{figure}[!htbp]
        \centering
        \includegraphics[width=3cm,height=3cm]{figures/g3.png}
        \caption{Pemilihan User}
        \label{User}
        \end{figure}

    \item Kemudian memilih tipe instalasi
    \begin{figure}[!htbp]
        \centering
        \includegraphics[width=3cm,height=3cm]{figures/g4.png}
        \caption{Pemilihan Direktori Penyimpanan}
        \label{Directory}
        \end{figure}

    \item Kemudian pilih lokasi untuk instal, setelah itu klik Next
    \begin{figure}[!htbp]
        \centering
        \includegraphics[width=3cm,height=3cm]{figures/g5.png}
        \caption{Pemilihan Opsi}
        \label{opsi}
        \end{figure}

    \item Kemudian centang pada kotak register dan pilih instal
    \begin{figure}[!htbp]
        \centering
        \includegraphics[width=3cm,height=3cm]{figures/g6.png}
        \caption{Proses Instal}
        \label{Proses}
        \end{figure}

    \item Kemudian Klik next
    \begin{figure}[!htbp]
        \centering
        \includegraphics[width=3cm,height=3cm]{figures/g7.png}
        \caption{Proses Instal Selesai}
        \label{Proses}
        \end{figure}

    \item kemudian jika tidak akan mengistal VSCode, maka pilih skip
    \begin{figure}[!htbp]
        \centering
        \includegraphics[width=3cm,height=3cm]{figures/g9.png}
        \caption{Penawaran Instal MS VSC}
        \label{offering}
        \end{figure}
        \item kemudian Proses intalasi telah selesai pilih finish
    \begin{figure}[!htbp]
        \centering
        \includegraphics[width=3cm,height=3cm]{figures/g12.png}
        \caption{Penawaran Instal MS VSC}
        \label{offering}
        \end{figure}
\end{enumerate}
\subsection(Spyder)
Spyder adalah tools yang dipakai untuk python
    \begin{figure}[!htbp]
        \centering
        \includegraphics[width=3cm,height=3cm]{figures/Screenshot(1).png}
        \caption{Instalasi Selesai}
        \label{akhir}
        \end{figure}
%%%%%%%%%%%%%%%%%%%%%%%%%%%%%%%%%%%%%%%%%%%%%%%%%%%%%%%%%%%%%%%%%%%%%%%%%%%%%%%%%%%%%%%%%%%%
\section{Arjun Yuda Firwanda}
\subsection{Sejarah Python}
Bahasa Python merupakan Bahasa pemrograman interpreatif multiguna dengan memanfaatkan perkembangan Bahasa pemrograman yang lebih fokus meningkatkan keterbacaan sebuah kode. Python digunakan sebagai Bahasa pemrograman yang menggabungkan kapabilitas, kemampuan, sintak kode yang sangat jelas.

Salah satu fitur yang tersedia pada bahasa python sebagai bahasa pemrograman dinamis yang dilengkapi dengan manajemen memori otomatis.

Saat ini kode python dapat dijalankan di berbagai platform sistem operasi, beberapa diantaranya Linux/Unix, Windows, Mac OS X, Java Virtual Machine, OS/2, Amiga, Palm.

Perkembangan Bahasa python digunakan untuk keperluan pengembangan perangkat lunak dan dapat berjalan diberbagai platform sistem operasi.
Bahasa pemrograman Python awalnya dikembangkan Guido van Rossum pada tahun 1990 di CWI, Amsterdam. Versi terakhir yang dikeluarkan CWI adalah 1.2 yang saat ini banyak digunakan oleh berbagai perusahaan di dunia.

\subsection{Instalasi Anaconda}
\begin{enumerate}
    \item Langkah pertama harus install Anaconda di Laptop/Pc anda.
    \item Jika yang belum download, silahkan download di google.
    \item Klik kanan pada unduhan Anaconda, klik kanan, pilih Run As Administrator.
    \item Klik next
    \begin{figure}[!htbp]
        \centering
        \includegraphics[width=3cm,height=3cm]{figures/gb1.png}
        \caption{File Instalasi Anaconda}
        \label{File}
        \end{figure}

    \item Klik Next
    \begin{figure}[!htbp]
        \centering
        \includegraphics[width=3cm,height=3cm]{figures/gb2.png}
        \caption{Tampilan Awal Instalasi Anconda}
        \label{Awal}
        \end{figure}

    \item Klik I Agree
    \begin{figure}[!htbp]
        \centering
        \includegraphics[width=3cm,height=3cm]{figures/gb3.png}
        \caption{Tampilan Licence Anaconda}
        \label{Licence}
        \end{figure}

    \item Klik Just Me dan Klik Next
    \begin{figure}[!htbp]
        \centering
        \includegraphics[width=3cm,height=3cm]{figures/gb4.png}
        \caption{Memilih Type Instalasi}
        \label{Type}
        \end{figure}

    \item Klik Next
    \begin{figure}[!htbp]
        \centering
        \includegraphics[width=3cm,height=3cm]{figures/gb5.png}
        \caption{Penyimpanan Instalasi Anaconda}
        \label{Penyimpanan}
        \end{figure}

    \item Pilih Semua Kemudian Klik Next
    \begin{figure}[!htbp]
        \centering
        \includegraphics[width=3cm,height=3cm]{figures/gb6.png}
        \caption{Opsi Anaconda}
        \label{Opsi}
        \end{figure}

    \item Menunggu Instalasi
    \begin{figure}[!htbp]
        \centering
        \includegraphics[width=3cm,height=3cm]{figures/gb7.png}
        \caption{Proses Installing}
        \label{Proses}
        \end{figure}

    \item Setelah Completed Klik Next
    \begin{figure}[!htbp]
        \centering
        \includegraphics[width=3cm,height=3cm]{figures/gb8.png}
        \caption{Installing Completed}
        \label{Completed}
        \end{figure}

    \item Setelah Completed Klik Instal Microsoft VSCode
    \begin{figure}[!htbp]
        \centering
        \includegraphics[width=3cm,height=3cm]{figures/gb9.png}
        \caption{Hak Akses Anaconda}
        \label{Akses}
        \end{figure}

    \item Klik Finish
    \begin{figure}[!htbp]
        \centering
        \includegraphics[width=3cm,height=3cm]{figures/gb10.png}
        \caption{Instalasi Selesai}
        \label{Selesai}
        \end{figure}
\end{enumerate}
\subsection{Spyder Python}
Spyder merupakan sebuah tools atau media yang dipakai dalam bahasa python.
    \begin{figure}[!htbp]
        \centering
        \includegraphics[width=3cm,height=3cm]{figures/gbrspyder.png}
        \caption{Spyder Python}
        \label{Spyder}
        \end{figure}
%%%%%%%%%%%%%%%%%%%%%%%%%%%%%%%%%%%%%%%%%%%%%%%%%%%%%%%%%%%%%%%%%%%%%%%%%%%%%%%%%%%%%%%%%%%%%%%%

\section{muhammad dzihan al-banna}
Python dikembangkan oleh Guido van Rossum pada tahun 1990 di CWI, Amsterdam, Belanda sebagai pengembangan dari bahasa pemrograman ABC.
Pada Tahun 1995, Guido pindah ke CNRI dan melanjutkan pengembangan Python. Saat ini pengembangan Python terus dilakukan oleh sekumpulan develop yang dipimpin oleh Guido dan Python Software Foundation. Python Software Foundation adalah sebuah organisasi non-profit yang dibentuk sebagai pemegang hak cipta intelektual Python sejak versi 2.1 dan dengan demikian mencegah Python dimiliki oleh perusahaan komersial. Saat ini distribusi Python sudah mencapai versi 2.7 dan versi 3.7

Nama Python diambil oleh Guido sebagai nama bahasa ciptaannya karena kecintaan Guido pada acara televisi Monty Python’s Flying Circus. Oleh karena itu seringkali ungkapan-ungkapan khas dari acara tersebut seringkali muncul dalam korespondensi antar pengguna Python.
\subsection{Instalasi Anaconda}
\begin{enumerate}
    \item Instal python terlebih dahulu.
    \item setelah itu download anaconda.
    \item Kemudian lakukan instalasi dan ikuti langkah-langkahnya.
    \item Klik next
    \begin{figure}[!htbp]
        \centering
        \includegraphics[width=3cm,height=3cm]{figures/dua.png}
        \caption{Instal file}
        \label{awal}
        \end{figure}

    \item Kemudian Klik I Agree
    \begin{figure}[!htbp]
        \centering
        \includegraphics[width=3cm,height=3cm]{figures/tiga.png}
        \caption{Lisensi}
        \label{License}
        \end{figure}

    \item Kemudian pilih akan di instal untuk siapa, kemudian pilih next
    \begin{figure}[!htbp]
        \centering
        \includegraphics[width=3cm,height=3cm]{figures/empat.png}
        \caption{Pemilihan User}
        \label{User}
        \end{figure}

    \item Kemudian tentukan dicretory nya, secara default akan berada di C
    \begin{figure}[!htbp]
        \centering
        \includegraphics[width=3cm,height=3cm]{figures/lima.png}
        \caption{Pemilihan Direktori Penyimpanan}
        \label{Directory}
        \end{figure}

    \item Kemudian Centang yang register Anaconda as default Python, Kemudian Pilih Next
    \begin{figure}[!htbp]
        \centering
        \includegraphics[width=3cm,height=3cm]{figures/enam.png}
        \caption{Pemilihan Opsi}
        \label{opsi}
        \end{figure}

    \item Tunggu Proses Instalasi hingga selesai
    \begin{figure}[!htbp]
        \centering
        \includegraphics[width=3cm,height=3cm]{figures/tujuh.png}
        \caption{Proses Instalasi}
        \label{Proses}
        \end{figure}

    \item Klik next
    \begin{figure}[!htbp]
        \centering
        \includegraphics[width=3cm,height=3cm]{figures/delapan.png}
        \caption{Instal selesai}
        \label{Proses}
        \end{figure}

    \item kemudian jika kalian belum instal MS VSC di sarankan menginstalnya dlu, jika sudah klik skip
    \begin{figure}[!htbp]
        \centering
        \includegraphics[width=3cm,height=3cm]{figures/sembilan.png}
        \caption{Penawaran Instal MS VSC}
        \label{offering}
        \end{figure}

    \item Instalasi anaconda telah selesai
    \begin{figure}[!htbp]
        \centering
        \includegraphics[width=3cm,height=3cm]{figures/sepuluh.png}
        \caption{Instalasi Selesai}
        \label{akhir}
        \end{figure}
\end{enumerate}
\subsection{Menggunakan Spyder}
Merupakan tool bawaan dari anaconda

%%%%%%%%%%%%%%%%%%%%%%%%%%%%%%%%%%%%%%%%%%%%%%%%%%%%%%%%%%

\section{Muhammad Fahmi}

\subsection{Sejarah Pyton}
Guido Van Rossum adalah orang yang menciptakan python di Scitching Mathematisch Centrum (CWI) Belanda pada wal tahun 1990-an.
Pada nama python sendiri berasal dari nama ular yang kita tau. Guido Van Rossum seorang yang menggemarkan grub komedi inggris yang bernama Monty Python, Lalu ia menamakan ciptaannya dengan nama Python
Bahasa python terinspirasi dari bahasa pemrograman ABC. Nama python tidak berasal dari nama ular yang kita kenal. Guido merupakan penggemar grup komedi Inggris bernama Monty Python. Kemudian, ia menamakan Bahasa pemrograman ciptaannya dengan nama Python.
Pada tahun 1994, Python 1.0 dirilis, yang diikuti dengan Python 2.0 pada tahun 2000. Python 3.0 keluar pada tahun 2008.

\subsection{Instalasi Anaconda}
\begin{enumerate}
    \item Pastikan Bahwa Python telah terinstall dilaptop anda.
    \item Download File Installer Anaconda pada www.anaconda.com
    \item Install seperti biasa
    \item Kemudian Klik I Agree
    \item Kemudian Centang yang register Anaconda as default Python, Kemudian Pilih Next
    \item Tunggu Proses Instalasi hingga selesai
    \begin{figure}[H]
		\includegraphics[width=10cm]{figures/fahmi/1.png}
		\centering
	\end{figure}

    \item Instalasi telah selesai
	 \begin{figure}[H]
		\includegraphics[width=10cm]{figures/fahmi/2.png}
		\centering
	\end{figure}
	
\end{enumerate}
	

\subsection{Penggunaan Spider}
Spyder merupakan text editor dari anaconda dimana di tools ini bisa menjalankan perintah-perintah dengan klik run saja.
Di anaconda sendiri mempunyai beberapa text editor untuk python tapi yang akan dijelaskan kali ini menggunakan spyder.

Spider sendiri sudah terinstall bersama pada saat kita menginstall Anacoda tadi.

ini adalah contoh tampilan pada Spider.
\begin{figure}[H]
		\includegraphics[width=10cm]{figures/fahmi/3.png}
		\centering
	\end{figure}
%%%%%%%%%%%%%%%%%%%%%%%%%%%%%%%%%%%%%%%%%%%%%%%%%%%%%%%%%%%%%%%%%%%%%%%%%%%%%%%%%%%%%%%%%%%%
\section{Evietania Charis Sujadi}

\section{Python}
    \subsection{Background}
    \label{Background}
    \par
    Python adalah sebuah bahasa pemrograman dengan level tinggi yang interaktif, dan mendukung berbagai paradigma pemrograman. Python sudah terkenal pada kalangan programmer sebagai bahasa yang mudah dipahami dan memiliki kompleksitas yang dinamis sehingga dapat dipakai di algoritma maupun platform yang berbagai macam.Python sudah memiliki banyak komunitas pendukung karena penggunanya yang banyak. Selain pada komunitas biasa, Python sudah diimplementasikan pada banyak perusahaan ternama dan dipasang pada aplikasi yang sudah terkenal seperti pada search engine google yang dimiliki oleh perusahaan Google.
    \par
    Python memiliki kepustakaan atau biasa disebut library yang sangat luas, dan dalam distribusi Python yang telah disediakan, hal tersebut diakibatkan oleh pendistribusian Python yang bebas karena bahasa pemrograman Python merupakan bahasa pemrograman yang freeware atau bebas dalam hal pengembangannya. Python adalah sebuah bahasa pemrograman yang dapat dengan mudah dibaca dan terstruktur, hal tersebut dikarenakan penggunaan sistem identasi, yaitu pemisahan blok-blok program susunan identasi, jadi untuk menambahkan sub-sub program dalam sebuah blok program, sub program tersebut harus diletakkan pada satu atau lebih spasi dari kolom sebuah blok
    \subsection{Problems}
        \begin{itemize}
            \item Kurangnya pemahaman tentang bahasa pemrograman Python
            \item Kurang mengerti dalam hal fungsi-fungsi yang terdapat pada bahasa pemrograman Python
        \end{itemize}

    \subsection{Objective and Contribution}
        \subsubsection{Objective}
            \begin{itemize}
                \item Dapat memahami tentang bahasa pemrograman Python
                \item Dapat memahami fungsi fungsi yang terdapat pada bahasa pemrograman Python
            \end{itemize}

        \subsubsection{Contribution}
            \begin{itemize}
                \item Dapat membangun sebuah sistem dengan menggunakan bahasa pemrograman Python
                \item Dapat membangun sebuah alat yang berguna, menggunakan mikrokontroler dan bahasa pemrograman python
            \end{itemize}

    \subsection{Scoop and Environtment}
        \begin{itemize}
            \item Pengenanalan tentang bahasa pemrograman Python
            \item Pengenalan fungsi-fungsi yang terdapat pada bahasa pemrograman Python
        \end{itemize}
%%%%%%%%%%%%%%%%%%%%%%%%%%%%%%%%%%%%%%%%%%%%%%%%%%%%%%%%%%%%%%%%%%%%%%%%%%%
\section{Habib Abdul Rasyid}
\subsection{Sejarah Python}
Python adalah salah satu bahasa pemrograman yang paling portabel, nyaman dan kuat yang tersedia saat ini. Ini juga didistribusikan secara bebas sebagai kode sumber yang dapat dimodifikasi dan didistribusikan kembali dalam produk komersial. Setiap programmer layak menikmati Python, dan Pemrograman Internet dengan Python menunjukkan caranya. Watters, van Rossum (penulis utama Python), dan James C. Ahlstrom mendemonstrasikan metode pemrograman dalam Python, dengan penekanan khusus pada aplikasi terkait internet\cite{watters1996internet}.
\par
\subsection{Instalasi Anaconda}
\begin{enumerate}
    \item Pastikan Bahwa Python telah terinstal dilaptop anda.
    \item Kemudian Download Anaconda pada websitenya.
    \item Buka installer yang telah di download barusan
    \item Klik next
     \begin{figure}[!htbp]
        \centering
        \includegraphics[width=3cm,height=3cm]{figures/habib/ss1.png}
        \caption{instalasi}
        \label{instalasi}
        \end{figure}
    \item klik I Agree
    \begin{figure}[!htbp]
        \centering
        \includegraphics[width=3cm,height=3cm]{figures/habib/ss2.png}
        \caption{instalasi}
        \label{instalasi}
        \end{figure}
    \item Klik Next
    \begin{figure}[!htbp]
        \centering
        \includegraphics[width=3cm,height=3cm]{figures/habib/ss3.png}
        \caption{instalasi}
        \label{instalasi}
        \end{figure}
    \item klik Install
    \begin{figure}[!htbp]
        \centering
        \includegraphics[width=3cm,height=3cm]{figures/habib/ss4.png}
        \caption{instalasi}
        \label{instalasi}
        \end{figure}
    \item Setelah Completed klik next
    \begin{figure}[!htbp]
        \centering
        \includegraphics[width=3cm,height=3cm]{figures/habib/ss5.png}
        \caption{instalasi}
        \label{instalasi}
        \end{figure}
    \item jika anda sudah mengintall Visual Studio code anda tinggal klik Skip saja.
    \begin{figure}[!htbp]
        \centering
        \includegraphics[width=3cm,height=3cm]{figures/habib/ss6.png}
        \caption{instalasi}
        \label{instalasi}
        \end{figure}
    \item Terakhir tinggal finish
    \begin{figure}[!htbp]
        \centering
        \includegraphics[width=3cm,height=3cm]{figures/habib/ss7.png}
        \caption{instalasi}
        \label{instalasi}
        \end{figure}
\end{enumerate}
\subsection{Spyder}
Setelah install anaconda didalamnya terdapat tools yang bisa di gunakan pada sesi ini menggunakan tool spyder

