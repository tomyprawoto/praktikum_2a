\section{Harun Ar - Rasyid}
\subsection{Teori}
\begin{enumerate}
    \item sebutkan jenis-jenis variabel dan jelaskan cara pemakaian variabel tersebut di
    kode Python
    Variabel merupakan tempat menyimpan data. Dalam phyton kita dapat membuat variable dengan cara sebagai berikut
    \lstinputlisting[firstline=74, lastline=76]{src/1174027.py}

    \item tuliskan bagaimana kode untuk meminta input dari user dan tuliskan bagaimana
    melakukan output ke layar.
    \lstinputlisting[firstline=78, lastline=79]{src/1174027.py}

    \item Tuliskan operator dasar aritmatika, tambah, kali, kurang bagi, dan bagaimana
    mengubah string ke integer dan integer ke string
    Operator  aritmatika adalah operator yang digunakan untuk melakukan perhitungan
    \lstinputlisting[firstline=81, lastline=90]{src/1174027.py}

    \item Tuliskan dan jelaskan sintak untuk perulangan, jenis-jenisnya contoh kode dan
    cara pakainya di python
    Untuk Perulangan Pada Python ada For dan While, Untuk Contohnya bisa lihat gambar berikut :
    \lstinputlisting[firstline=92, lastline=98]{src/1174027.py}

    \item Tuliskan jelaskan cara pakai sintak untuk memilih kondisi, dan bagiamana con-
    toh sintak kondisi di dalam kondisi.
    Pengambilan kondisi If yang digunakan untuk mengantisipasi kondisi yang terjadi saat program dijalankan dan menentukan tindakan apa yang akan diambil sesuai dengan kondisi.
    If statement
    \lstinputlisting[firstline=100, lastline=102]{src/1174027.py}
    
    Ifelse
    \lstinputlisting[firstline=104, lastline=107]{src/1174027.py}
    
    IfNested
    \lstinputlisting[firstline=109, lastline=114]{src/1174027.py}

    \item Tuliskan apa saja jenis error yang sering ditemui di python dalam mengerjakan
    sintak diatas. dan bagaimana cara mengatasinya
    \begin{itemize}
        \item Exception
        Kelas dasar untuk semua pengecualian / exception

        \item Stoplteration
        Dibesarkan ketika metode (iterator) berikutnya dari iterator tidak mengarah ke objek apa pun.

        \item SystemExit
        Dibesarkan oleh fungsi sys.exit ().

        \item StandardError
        Kelas dasar untuk semua pengecualian built-in kecuali StopIteration dan SystemExit.

        \item ArithmeticError
        Kelas dasar untuk semua kesalahan yang terjadi untuk perhitungan numerik.

        \item OverflowError
        Dibesarkan saat perhitungan melebihi batas maksimum untuk tipe numerik.

        \item FloatingPointError
        Dibesarkan saat perhitungan floating point gagal.

        \item ZeroDivisonError
        Dibesarkan saat pembagian atau modulo nol dilakukan untuk semua tipe numerik.

        \item AssertionError
        Dibesarkan jika terjadi kegagalan pernyataan Assert.

        \item AttributeError
        Dibesarkan jika terjadi kegagalan referensi atribut atau penugasan.
         
        \item EOFError
        Dibesarkan bila tidak ada input dari fungsi rawinput () atau input () dan akhir file tercapai.

        \item ImportError
        Dibesarkan saat sebuah pernyataan impor gagal.

        \item KeyboardInterrupt
        Dibesarkan saat pengguna menyela eksekusi program, biasanya dengan menekan Ctrl + c.

        \item LookupError
        Kelas dasar untuk semua kesalahan pencarian.

        \item IndexError
        Dibesarkan saat sebuah indeks tidak ditemukan secara berurutan.

        \item KeyError
        Dibesarkan saat kunci yang ditentukan tidak ditemukan dalam kamus.

        \item NameError
        Dibesarkan saat pengenal tidak ditemukan di namespace lokal atau global.

        \item UnboundLocalError
        Dibesarkan saat mencoba mengakses variabel lokal dalam suatu fungsi atau metode namun tidak ada nilai yang ditugaskan padanya.

        \item EnvironmentError
        Kelas dasar untuk semua pengecualian yang terjadi di luar lingkungan Python.

        \item IOError
        Dibesarkan saat operasi input / output gagal, seperti pernyataan cetak atau fungsi open () saat mencoba membuka file yang tidak ada.

        \item OSError
        Dibangkitkan untuk kesalahan terkait sistem operasi.

        \item SyntaxError
        Dibesarkan saat ada kesalahan dengan sintaks Python.

        \item IndentationError
        Dibesarkan saat indentasi tidak ditentukan dengan benar.

        \item SystemError
        Dibesarkan saat penafsir menemukan masalah internal, namun bila kesalahan ini ditemui juru bahasa Python tidak keluar.

        \item SystemExit
        Dibesarkan saat juru bahasa Python berhenti dengan menggunakan fungsi sys.exit (). Jika tidak ditangani dalam kode, menyebabkan penafsir untuk keluar.

        \item TypeError
        Dibesarkan saat operasi atau fungsi dicoba yang tidak valid untuk tipe data yang ditentukan.

        \item ValueError
        Dibesarkan ketika fungsi bawaan untuk tipe data memiliki jenis argumen yang valid, namun argumen tersebut memiliki nilai yang tidak valid yang ditentukan.

        \item RuntimeError
        Dibesarkan saat kesalahan yang dihasilkan tidak termasuk dalam kategori apa pun.

        \item NotImplementedError
        Dibesarkan ketika metode abstrak yang perlu diimplementasikan di kelas warisan sebenarnya tidak dilaksanakan.
    \end{itemize}

    \item Tuliskan dan jelaskan cara memakai Try Except.
    \lstinputlisting[firstline=116, lastline=122]{src/1174027.py}

\end{enumerate}
\subsection{Ketrampilan Pemrograman}
\begin{enumerate}
    \item Buatlah luaran huruf yang dirangkai dari tanda bintang, pagar atau plus dari
    NPM kita. Tanda bintang untuk NPM mod 3=0, tanda pagar untuk NPM mod
    3 =1, tanda plus untuk NPM mod3=2.
    \lstinputlisting[firstline=7, lastline=17]{src/1174027.py}

    \item Buatlah program hello word dengan input NPM yang disimpan dalam sebuah
    variabel string bernama NPM dan output sebanyak dua dijit belakang NPM.
    \lstinputlisting[firstline=19, lastline=24]{src/1174027.py}
    
    \item Buatlah program hello word dengan input nama yang disimpan dalam sebuah
    variabel string bernama NPM dan beri luaran output berupa tiga karakter
    belakang dari NPM sebanyak penjumlahan tiga dijit tersebut.
    \lstinputlisting[firstline=26, lastline=31]{src/1174027.py}

    \item Buatlah program hello word dengan input nama yang disimpan dalam sebuah
    variabel string bernama NPM dan beri luaran output berupa digit ketiga dari
    belakang dari variabel NPM,
    \lstinputlisting[firstline=33, lastline=35]{src/1174027.py}

    \item Buat program dengan mengisi variabel alfabet
    dengan nomor npm satu persatu berurut.
    \lstinputlisting[firstline=37, lastline=48]{src/1174027.py}

    \item Dari soal no 5, Lakukan penjumlahan dari seluruh variabel tersebut.
    \lstinputlisting[firstline=50, lastline=51]{src/1174027.py}

    \item Dari soal no 5, Lakukan perkalian dari seluruh variabel tersebut.
    \lstinputlisting[firstline=53, lastline=54]{src/1174027.py}

    \item Dari soal no 5, Lakukan print secara vertikal dari NPM anda menggunakan
    variabel diatas.
    \lstinputlisting[firstline=56, lastline=63]{src/1174027.py}

    \item Dari soal no 5, Lakukan print NPM anda tapi hanya dijit genap saja.
    \lstinputlisting[firstline=65, lastline=66]{src/1174027.py}

    \item Dari soal no 5, Lakukan print NPM anda tapi hanya dijit ganjil saja.
    \lstinputlisting[firstline=68, lastline=69]{src/1174027.py}

    \item Dari soal no 5, Lakukan print NPM anda tapi hanya dijit yang termasuk bilan-
    gan prima saja.
    \lstinputlisting[firstline=71, lastline=72]{src/1174027.py}

\end{enumerate}
\subsection{Ketrampilan Penanganan Error}
    \lstinputlisting[language=Python]{src/err2.py}